% Options for packages loaded elsewhere
\PassOptionsToPackage{unicode}{hyperref}
\PassOptionsToPackage{hyphens}{url}
%
\documentclass[
]{article}
\usepackage{lmodern}
\usepackage{amssymb,amsmath}
\usepackage{ifxetex,ifluatex}
\ifnum 0\ifxetex 1\fi\ifluatex 1\fi=0 % if pdftex
  \usepackage[T1]{fontenc}
  \usepackage[utf8]{inputenc}
  \usepackage{textcomp} % provide euro and other symbols
\else % if luatex or xetex
  \usepackage{unicode-math}
  \defaultfontfeatures{Scale=MatchLowercase}
  \defaultfontfeatures[\rmfamily]{Ligatures=TeX,Scale=1}
\fi
% Use upquote if available, for straight quotes in verbatim environments
\IfFileExists{upquote.sty}{\usepackage{upquote}}{}
\IfFileExists{microtype.sty}{% use microtype if available
  \usepackage[]{microtype}
  \UseMicrotypeSet[protrusion]{basicmath} % disable protrusion for tt fonts
}{}
\makeatletter
\@ifundefined{KOMAClassName}{% if non-KOMA class
  \IfFileExists{parskip.sty}{%
    \usepackage{parskip}
  }{% else
    \setlength{\parindent}{0pt}
    \setlength{\parskip}{6pt plus 2pt minus 1pt}}
}{% if KOMA class
  \KOMAoptions{parskip=half}}
\makeatother
\usepackage{xcolor}
\IfFileExists{xurl.sty}{\usepackage{xurl}}{} % add URL line breaks if available
\IfFileExists{bookmark.sty}{\usepackage{bookmark}}{\usepackage{hyperref}}
\hypersetup{
  pdftitle={Tegaderm CHG IV Securement Dressing for Central Venous and Arterial Catheter Insertion Sites},
  pdfauthor={Andrew J. Sims},
  hidelinks,
  pdfcreator={LaTeX via pandoc}}
\urlstyle{same} % disable monospaced font for URLs
\usepackage[margin=1in]{geometry}
\usepackage{color}
\usepackage{fancyvrb}
\newcommand{\VerbBar}{|}
\newcommand{\VERB}{\Verb[commandchars=\\\{\}]}
\DefineVerbatimEnvironment{Highlighting}{Verbatim}{commandchars=\\\{\}}
% Add ',fontsize=\small' for more characters per line
\usepackage{framed}
\definecolor{shadecolor}{RGB}{248,248,248}
\newenvironment{Shaded}{\begin{snugshade}}{\end{snugshade}}
\newcommand{\AlertTok}[1]{\textcolor[rgb]{0.94,0.16,0.16}{#1}}
\newcommand{\AnnotationTok}[1]{\textcolor[rgb]{0.56,0.35,0.01}{\textbf{\textit{#1}}}}
\newcommand{\AttributeTok}[1]{\textcolor[rgb]{0.77,0.63,0.00}{#1}}
\newcommand{\BaseNTok}[1]{\textcolor[rgb]{0.00,0.00,0.81}{#1}}
\newcommand{\BuiltInTok}[1]{#1}
\newcommand{\CharTok}[1]{\textcolor[rgb]{0.31,0.60,0.02}{#1}}
\newcommand{\CommentTok}[1]{\textcolor[rgb]{0.56,0.35,0.01}{\textit{#1}}}
\newcommand{\CommentVarTok}[1]{\textcolor[rgb]{0.56,0.35,0.01}{\textbf{\textit{#1}}}}
\newcommand{\ConstantTok}[1]{\textcolor[rgb]{0.00,0.00,0.00}{#1}}
\newcommand{\ControlFlowTok}[1]{\textcolor[rgb]{0.13,0.29,0.53}{\textbf{#1}}}
\newcommand{\DataTypeTok}[1]{\textcolor[rgb]{0.13,0.29,0.53}{#1}}
\newcommand{\DecValTok}[1]{\textcolor[rgb]{0.00,0.00,0.81}{#1}}
\newcommand{\DocumentationTok}[1]{\textcolor[rgb]{0.56,0.35,0.01}{\textbf{\textit{#1}}}}
\newcommand{\ErrorTok}[1]{\textcolor[rgb]{0.64,0.00,0.00}{\textbf{#1}}}
\newcommand{\ExtensionTok}[1]{#1}
\newcommand{\FloatTok}[1]{\textcolor[rgb]{0.00,0.00,0.81}{#1}}
\newcommand{\FunctionTok}[1]{\textcolor[rgb]{0.00,0.00,0.00}{#1}}
\newcommand{\ImportTok}[1]{#1}
\newcommand{\InformationTok}[1]{\textcolor[rgb]{0.56,0.35,0.01}{\textbf{\textit{#1}}}}
\newcommand{\KeywordTok}[1]{\textcolor[rgb]{0.13,0.29,0.53}{\textbf{#1}}}
\newcommand{\NormalTok}[1]{#1}
\newcommand{\OperatorTok}[1]{\textcolor[rgb]{0.81,0.36,0.00}{\textbf{#1}}}
\newcommand{\OtherTok}[1]{\textcolor[rgb]{0.56,0.35,0.01}{#1}}
\newcommand{\PreprocessorTok}[1]{\textcolor[rgb]{0.56,0.35,0.01}{\textit{#1}}}
\newcommand{\RegionMarkerTok}[1]{#1}
\newcommand{\SpecialCharTok}[1]{\textcolor[rgb]{0.00,0.00,0.00}{#1}}
\newcommand{\SpecialStringTok}[1]{\textcolor[rgb]{0.31,0.60,0.02}{#1}}
\newcommand{\StringTok}[1]{\textcolor[rgb]{0.31,0.60,0.02}{#1}}
\newcommand{\VariableTok}[1]{\textcolor[rgb]{0.00,0.00,0.00}{#1}}
\newcommand{\VerbatimStringTok}[1]{\textcolor[rgb]{0.31,0.60,0.02}{#1}}
\newcommand{\WarningTok}[1]{\textcolor[rgb]{0.56,0.35,0.01}{\textbf{\textit{#1}}}}
\usepackage{longtable,booktabs}
% Correct order of tables after \paragraph or \subparagraph
\usepackage{etoolbox}
\makeatletter
\patchcmd\longtable{\par}{\if@noskipsec\mbox{}\fi\par}{}{}
\makeatother
% Allow footnotes in longtable head/foot
\IfFileExists{footnotehyper.sty}{\usepackage{footnotehyper}}{\usepackage{footnote}}
\makesavenoteenv{longtable}
\usepackage{graphicx,grffile}
\makeatletter
\def\maxwidth{\ifdim\Gin@nat@width>\linewidth\linewidth\else\Gin@nat@width\fi}
\def\maxheight{\ifdim\Gin@nat@height>\textheight\textheight\else\Gin@nat@height\fi}
\makeatother
% Scale images if necessary, so that they will not overflow the page
% margins by default, and it is still possible to overwrite the defaults
% using explicit options in \includegraphics[width, height, ...]{}
\setkeys{Gin}{width=\maxwidth,height=\maxheight,keepaspectratio}
% Set default figure placement to htbp
\makeatletter
\def\fps@figure{htbp}
\makeatother
\setlength{\emergencystretch}{3em} % prevent overfull lines
\providecommand{\tightlist}{%
  \setlength{\itemsep}{0pt}\setlength{\parskip}{0pt}}
\setcounter{secnumdepth}{-\maxdimen} % remove section numbering

\title{Tegaderm CHG IV Securement Dressing for Central Venous and Arterial
Catheter Insertion Sites}
\usepackage{etoolbox}
\makeatletter
\providecommand{\subtitle}[1]{% add subtitle to \maketitle
  \apptocmd{\@title}{\par {\large #1 \par}}{}{}
}
\makeatother
\subtitle{A decision tree example with probabilistic sensitivity analysis}
\author{Andrew J. Sims}
\date{2021-01-29}

\begin{document}
\maketitle

\hypertarget{introduction}{%
\section{Introduction}\label{introduction}}

This vignette is an example of modelling a decision tree using the
\texttt{rdecision} package, with probabilistic sensitivity analysis. It
is based on the model reported by Jenks \emph{et al} {[}1{]} in which a
transparent dressing used to secure vascular catheters (Tegaderm CHG)
was compared with a standard dressing.

Two methods of evaluating the decision are presented. The first method
constructs a decision tree and evaluates the costs associated with
traversing each pathway through it. The second method is a direct
calculation and summation of costs, without the need to construct a
tree. Point estimates and probabilistic sensitivity analysis are
conducted for both methods.

\hypertarget{model-variables}{%
\section{Model variables}\label{model-variables}}

Thirteen variables were used in the model. The choice of variables,
their distributions and their parameters are taken from table 3 of Jenks
\emph{et al} {[}1{]}, with the following corrections:

\begin{itemize}
\tightlist
\item
  For variables with lognormal uncertainty, the manufacturer gave values
  for the mean \(m\) and standard deviation \(s\) in log space. However,
  their standard deviations were quoted as negative values. This was an
  error, but had no effect on their results, because they sampled values
  of \(\exp(m + sz)\), where \(z\) is a sample from a standard normal
  distribution and is symmetrical about 0. For the variables with log
  normal uncertainty given below, positive standard deviation
  parameters, with the same absolute value, have been used as
  hyperparameters of the log normal distributions. Also note that the
  \emph{median} value on the natural scale of a random variable
  distributed as \(logN(m,s)\), where \(\mu\) and \(\sigma\) are the
  mean and standard deviation on the log scale, is \(e^\mu\); the mean
  on the natural scale is slightly larger. For example, the hazard ratio
  for CRBSI with Tegaderm versus standard dressing was modelled as
  \(logN(-0.911,0.393)\), which has median 0.402 (the point estimate of
  the ratio from the literature) and mean 0.434.
\item
  The relative risk for dermatitis was modelled as
  \(logN(1.482,0.490)\).
\item
  The point estimate cost of CRBSI was £9900, not £9990, although the
  parameters (198,50) are quoted correctly.
\end{itemize}

The 13 model variables were constructed as follows:

\begin{Shaded}
\begin{Highlighting}[]
\CommentTok{# clinical variables}
\NormalTok{r.CRBSI <-}\StringTok{ }\NormalTok{NormModVar}\OperatorTok{$}\KeywordTok{new}\NormalTok{(}
  \StringTok{'Baseline CRBSI rate'}\NormalTok{, }\StringTok{'/1000 catheter days'}\NormalTok{, }\DataTypeTok{mu=}\FloatTok{1.48}\NormalTok{, }\DataTypeTok{sigma=}\FloatTok{0.074}
\NormalTok{)}
\NormalTok{hr.CRBSI <-}\StringTok{ }\NormalTok{LogNormModVar}\OperatorTok{$}\KeywordTok{new}\NormalTok{(}
  \StringTok{'Tegaderm CRBSI HR'}\NormalTok{, }\StringTok{'ratio'}\NormalTok{, }\DataTypeTok{p1=}\OperatorTok{-}\FloatTok{0.911}\NormalTok{, }\DataTypeTok{p2=}\FloatTok{0.393}
\NormalTok{)}
\NormalTok{r.LSI <-}\StringTok{ }\NormalTok{NormModVar}\OperatorTok{$}\KeywordTok{new}\NormalTok{(}
  \StringTok{'Baseline LSI rate'}\NormalTok{, }\StringTok{'/patient'}\NormalTok{, }\DataTypeTok{mu=}\FloatTok{0.1}\NormalTok{, }\DataTypeTok{sigma=}\FloatTok{0.01}
\NormalTok{)}
\NormalTok{hr.LSI <-}\StringTok{ }\NormalTok{LogNormModVar}\OperatorTok{$}\KeywordTok{new}\NormalTok{(}
  \StringTok{'Tegaderm LSI HR'}\NormalTok{, }\StringTok{'ratio'}\NormalTok{, }\DataTypeTok{p1=}\OperatorTok{-}\FloatTok{0.911}\NormalTok{, }\DataTypeTok{p2=}\FloatTok{0.393}
\NormalTok{)}
\NormalTok{r.Dermatitis <-}\StringTok{ }\NormalTok{NormModVar}\OperatorTok{$}\KeywordTok{new}\NormalTok{(}
  \StringTok{'Baseline dermatitis risk'}\NormalTok{, }\StringTok{'/catheter'}\NormalTok{, }\DataTypeTok{mu=}\FloatTok{0.0026}\NormalTok{, }\DataTypeTok{sigma=}\FloatTok{0.00026}
\NormalTok{)}
\NormalTok{rr.Dermatitis <-}\StringTok{ }\NormalTok{LogNormModVar}\OperatorTok{$}\KeywordTok{new}\NormalTok{(}
  \StringTok{'Tegaderm Dermatitis RR'}\NormalTok{, }\StringTok{'ratio'}\NormalTok{,  }\DataTypeTok{p1=}\FloatTok{1.482}\NormalTok{, }\DataTypeTok{p2=}\FloatTok{0.490}
\NormalTok{)}

\CommentTok{# cost variables}
\NormalTok{c.CRBSI <-}\StringTok{ }\NormalTok{GammaModVar}\OperatorTok{$}\KeywordTok{new}\NormalTok{(}
  \StringTok{'CRBSI cost'}\NormalTok{, }\StringTok{'GBP'}\NormalTok{, }\DataTypeTok{shape=}\FloatTok{198.0}\NormalTok{, }\DataTypeTok{scale=}\DecValTok{50}
\NormalTok{)}
\NormalTok{c.Dermatitis <-}\StringTok{ }\NormalTok{GammaModVar}\OperatorTok{$}\KeywordTok{new}\NormalTok{(}
  \StringTok{'Dermatitis cost'}\NormalTok{, }\StringTok{'GBP'}\NormalTok{, }\DataTypeTok{shape=}\DecValTok{30}\NormalTok{, }\DataTypeTok{scale=}\DecValTok{5}
\NormalTok{)}
\NormalTok{c.LSI <-}\StringTok{ }\NormalTok{GammaModVar}\OperatorTok{$}\KeywordTok{new}\NormalTok{(}
    \StringTok{'LSI cost'}\NormalTok{, }\StringTok{'GBP'}\NormalTok{, }\DataTypeTok{shape=}\DecValTok{50}\NormalTok{, }\DataTypeTok{scale=}\DecValTok{5}
\NormalTok{  )}
\NormalTok{c.Tegaderm <-}\StringTok{ }\NormalTok{ConstModVar}\OperatorTok{$}\KeywordTok{new}\NormalTok{(}
  \StringTok{'Tegaderm CHG cost'}\NormalTok{, }\StringTok{'GBP'}\NormalTok{, }\DataTypeTok{const=}\FloatTok{6.21}
\NormalTok{)}
\NormalTok{c.Standard <-}\StringTok{ }\NormalTok{ConstModVar}\OperatorTok{$}\KeywordTok{new}\NormalTok{(}
  \StringTok{'Standard dressing cost'}\NormalTok{, }\StringTok{'GBP'}\NormalTok{, }\DataTypeTok{const=}\FloatTok{1.34}
\NormalTok{)}
\NormalTok{n.cathdays <-}\StringTok{ }\NormalTok{NormModVar}\OperatorTok{$}\KeywordTok{new}\NormalTok{(}
  \StringTok{'No. days with catheter'}\NormalTok{, }\StringTok{'days'}\NormalTok{, }\DataTypeTok{mu=}\DecValTok{10}\NormalTok{, }\DataTypeTok{sigma=}\DecValTok{2}
\NormalTok{)}
\NormalTok{n.dressings <-}\StringTok{ }\NormalTok{NormModVar}\OperatorTok{$}\KeywordTok{new}\NormalTok{(}
  \StringTok{'No. dressings'}\NormalTok{, }\StringTok{'dressings'}\NormalTok{, }\DataTypeTok{mu=}\DecValTok{3}\NormalTok{, }\DataTypeTok{sigma=}\FloatTok{0.3}
\NormalTok{)}
\end{Highlighting}
\end{Shaded}

\hypertarget{the-decision-tree-approach}{%
\section{The decision tree approach}\label{the-decision-tree-approach}}

The decision problem may be solved by constructing a decision tree
comprising decision nodes, chance nodes and leaf nodes. The general
approach is to create expressions involving model variables, construct a
tree, and then evaluate the decision for the base case and its
uncertainty.

\hypertarget{model-variable-expressions}{%
\subsection{Model variable
expressions}\label{model-variable-expressions}}

Variables in the model may be included in the decision tree via
mathematical expressions, which involve model variables and are
themselves model variables. Forms of expression involving R's numerical
functions and multiple model variables are supported, provided they
conform to R syntax. The following code creates the model variable
expressions to be used as values in the decision tree edges.

\begin{Shaded}
\begin{Highlighting}[]
\CommentTok{# probabilities}
\NormalTok{p.Dermatitis.S <-}\StringTok{ }\NormalTok{ExprModVar}\OperatorTok{$}\KeywordTok{new}\NormalTok{(}
  \StringTok{'P(dermatitis|standard dressing)'}\NormalTok{, }\StringTok{'P'}\NormalTok{, }
\NormalTok{  rlang}\OperatorTok{::}\KeywordTok{quo}\NormalTok{(n.dressings}\OperatorTok{*}\NormalTok{r.Dermatitis)}
\NormalTok{)}
\NormalTok{p.Dermatitis.T <-}\StringTok{ }\NormalTok{ExprModVar}\OperatorTok{$}\KeywordTok{new}\NormalTok{(}
  \StringTok{'P(dermatitis|Tegaderm)'}\NormalTok{, }\StringTok{'P'}\NormalTok{, }
\NormalTok{  rlang}\OperatorTok{::}\KeywordTok{quo}\NormalTok{(n.dressings}\OperatorTok{*}\NormalTok{r.Dermatitis}\OperatorTok{*}\NormalTok{rr.Dermatitis)}
\NormalTok{)}
\NormalTok{r.LSI.T <-}\StringTok{ }\NormalTok{ExprModVar}\OperatorTok{$}\KeywordTok{new}\NormalTok{(}
  \StringTok{'P(LSI|Tegaderm)'}\NormalTok{, }\StringTok{'P'}\NormalTok{, rlang}\OperatorTok{::}\KeywordTok{quo}\NormalTok{(r.LSI}\OperatorTok{*}\NormalTok{hr.LSI)}
\NormalTok{)}
\NormalTok{p.CRBSI.S <-}\StringTok{ }\NormalTok{ExprModVar}\OperatorTok{$}\KeywordTok{new}\NormalTok{(}
  \StringTok{'P(CRBSI|standard dressing)'}\NormalTok{, }\StringTok{'P'}\NormalTok{,  rlang}\OperatorTok{::}\KeywordTok{quo}\NormalTok{(r.CRBSI}\OperatorTok{*}\NormalTok{n.cathdays}\OperatorTok{/}\DecValTok{1000}\NormalTok{)}
\NormalTok{)}
\NormalTok{p.CRBSI.T <-}\StringTok{ }\NormalTok{ExprModVar}\OperatorTok{$}\KeywordTok{new}\NormalTok{(}
  \StringTok{'P(CRBSI|Tegaderm)'}\NormalTok{, }\StringTok{'P'}\NormalTok{, rlang}\OperatorTok{::}\KeywordTok{quo}\NormalTok{(r.CRBSI}\OperatorTok{*}\NormalTok{n.cathdays}\OperatorTok{*}\NormalTok{hr.CRBSI}\OperatorTok{/}\DecValTok{1000}\NormalTok{)}
\NormalTok{)}
\NormalTok{p.NoComp.S <-}\StringTok{ }\NormalTok{ExprModVar}\OperatorTok{$}\KeywordTok{new}\NormalTok{(}\StringTok{"P(No comp|standard dressing)"}\NormalTok{, }\StringTok{"P"}\NormalTok{,}
\NormalTok{  rlang}\OperatorTok{::}\KeywordTok{quo}\NormalTok{(}\DecValTok{1}\OperatorTok{-}\NormalTok{(p.Dermatitis.S}\OperatorTok{+}\NormalTok{r.LSI}\OperatorTok{+}\NormalTok{p.CRBSI.S))}
\NormalTok{)}
\NormalTok{p.NoComp.T <-}\StringTok{ }\NormalTok{ExprModVar}\OperatorTok{$}\KeywordTok{new}\NormalTok{(}\StringTok{"P(No comp|Tegaderm)"}\NormalTok{, }\StringTok{"P"}\NormalTok{,}
\NormalTok{  rlang}\OperatorTok{::}\KeywordTok{quo}\NormalTok{(}\DecValTok{1}\OperatorTok{-}\NormalTok{(p.Dermatitis.T}\OperatorTok{+}\NormalTok{r.LSI.T}\OperatorTok{+}\NormalTok{p.CRBSI.T))}
\NormalTok{)}
\end{Highlighting}
\end{Shaded}

\hypertarget{the-decision-tree}{%
\subsection{The decision tree}\label{the-decision-tree}}

The following code constructs the decision tree based on figure 2 of
Jenks \emph{et al} {[}1{]}. In the formulation used by
\texttt{rdecision}, the decision tree is constructed from sets of
decision, chance and leaf nodes and from edges (actions and reactions).
Leaf nodes are synonymous with pathways in Briggs' terminology {[}2{]}).
The time horizon is not stated explicitly in the model, and is assumed
to be 7 days here. It was implied that the time horizon was ICU stay
plus some follow-up, and the costs reflect those incurred in that
period, so the assumption of 7 days does not affect the
\texttt{rdecision} implementation of the model.

\begin{Shaded}
\begin{Highlighting}[]
\CommentTok{# create decision tree}
\NormalTok{th <-}\StringTok{ }\KeywordTok{as.difftime}\NormalTok{(}\DecValTok{7}\NormalTok{, }\DataTypeTok{units=}\StringTok{"days"}\NormalTok{)}
\CommentTok{# standard dressing branch}
\NormalTok{t1 <-}\StringTok{ }\NormalTok{LeafNode}\OperatorTok{$}\KeywordTok{new}\NormalTok{(}\StringTok{"Dermatitis"}\NormalTok{, }\DataTypeTok{interval=}\NormalTok{th)}
\NormalTok{t2 <-}\StringTok{ }\NormalTok{LeafNode}\OperatorTok{$}\KeywordTok{new}\NormalTok{(}\StringTok{"LSI"}\NormalTok{, }\DataTypeTok{interval=}\NormalTok{th)}
\NormalTok{t3 <-}\StringTok{ }\NormalTok{LeafNode}\OperatorTok{$}\KeywordTok{new}\NormalTok{(}\StringTok{"CRBSI"}\NormalTok{, }\DataTypeTok{interval=}\NormalTok{th)}
\NormalTok{t4 <-}\StringTok{ }\NormalTok{LeafNode}\OperatorTok{$}\KeywordTok{new}\NormalTok{(}\StringTok{"No comp"}\NormalTok{, }\DataTypeTok{interval=}\NormalTok{th)}
\NormalTok{c1 <-}\StringTok{ }\NormalTok{ChanceNode}\OperatorTok{$}\KeywordTok{new}\NormalTok{()}
\NormalTok{e1 <-}\StringTok{ }\NormalTok{Reaction}\OperatorTok{$}\KeywordTok{new}\NormalTok{(c1,t1,}\DataTypeTok{p=}\NormalTok{p.Dermatitis.S,}\DataTypeTok{cost=}\NormalTok{c.Dermatitis)}
\NormalTok{e2 <-}\StringTok{ }\NormalTok{Reaction}\OperatorTok{$}\KeywordTok{new}\NormalTok{(c1,t2,}\DataTypeTok{p=}\NormalTok{r.LSI,}\DataTypeTok{cost=}\NormalTok{c.LSI)}
\NormalTok{e3 <-}\StringTok{ }\NormalTok{Reaction}\OperatorTok{$}\KeywordTok{new}\NormalTok{(c1,t3,}\DataTypeTok{p=}\NormalTok{p.CRBSI.S,}\DataTypeTok{cost=}\NormalTok{c.CRBSI)}
\NormalTok{e4 <-}\StringTok{ }\NormalTok{Reaction}\OperatorTok{$}\KeywordTok{new}\NormalTok{(c1,t4,}\DataTypeTok{p=}\NormalTok{p.NoComp.S,}\DataTypeTok{cost=}\DecValTok{0}\NormalTok{)}
\CommentTok{# Tegaderm dressing branch}
\NormalTok{t5 <-}\StringTok{ }\NormalTok{LeafNode}\OperatorTok{$}\KeywordTok{new}\NormalTok{(}\StringTok{"Dermatitis"}\NormalTok{, }\DataTypeTok{interval=}\NormalTok{th)}
\NormalTok{t6 <-}\StringTok{ }\NormalTok{LeafNode}\OperatorTok{$}\KeywordTok{new}\NormalTok{(}\StringTok{"LSI"}\NormalTok{, }\DataTypeTok{interval=}\NormalTok{th)}
\NormalTok{t7 <-}\StringTok{ }\NormalTok{LeafNode}\OperatorTok{$}\KeywordTok{new}\NormalTok{(}\StringTok{"CRBSI"}\NormalTok{, }\DataTypeTok{interval=}\NormalTok{th)}
\NormalTok{t8 <-}\StringTok{ }\NormalTok{LeafNode}\OperatorTok{$}\KeywordTok{new}\NormalTok{(}\StringTok{"No comp"}\NormalTok{, }\DataTypeTok{interval=}\NormalTok{th)}
\NormalTok{c2 <-}\StringTok{ }\NormalTok{ChanceNode}\OperatorTok{$}\KeywordTok{new}\NormalTok{()}
\NormalTok{e5 <-}\StringTok{ }\NormalTok{Reaction}\OperatorTok{$}\KeywordTok{new}\NormalTok{(c2,t5,}\DataTypeTok{p=}\NormalTok{p.Dermatitis.T,}\DataTypeTok{cost=}\NormalTok{c.Dermatitis)}
\NormalTok{e6 <-}\StringTok{ }\NormalTok{Reaction}\OperatorTok{$}\KeywordTok{new}\NormalTok{(c2,t6,}\DataTypeTok{p=}\NormalTok{r.LSI.T,}\DataTypeTok{cost=}\NormalTok{c.LSI)}
\NormalTok{e7 <-}\StringTok{ }\NormalTok{Reaction}\OperatorTok{$}\KeywordTok{new}\NormalTok{(c2,t7,}\DataTypeTok{p=}\NormalTok{p.CRBSI.T,}\DataTypeTok{cost=}\NormalTok{c.CRBSI)}
\NormalTok{e8 <-}\StringTok{ }\NormalTok{Reaction}\OperatorTok{$}\KeywordTok{new}\NormalTok{(c2,t8,}\DataTypeTok{p=}\NormalTok{p.NoComp.T,}\DataTypeTok{cost=}\DecValTok{0}\NormalTok{)}
\CommentTok{# decision node}
\NormalTok{d1 <-}\StringTok{ }\NormalTok{DecisionNode}\OperatorTok{$}\KeywordTok{new}\NormalTok{(}\StringTok{"d1"}\NormalTok{)}
\NormalTok{e9 <-}\StringTok{ }\NormalTok{Action}\OperatorTok{$}\KeywordTok{new}\NormalTok{(d1,c1,}\DataTypeTok{label=}\StringTok{"Standard"}\NormalTok{,}\DataTypeTok{cost=}\NormalTok{c.Standard)}
\NormalTok{e10 <-}\StringTok{ }\NormalTok{Action}\OperatorTok{$}\KeywordTok{new}\NormalTok{(d1,c2,}\DataTypeTok{label=}\StringTok{"Tegaderm"}\NormalTok{,}\DataTypeTok{cost=}\NormalTok{c.Tegaderm)}
\CommentTok{# create decision tree}
\NormalTok{V <-}\StringTok{ }\KeywordTok{list}\NormalTok{(d1,c1,c2,t1,t2,t3,t4,t5,t6,t7,t8)}
\NormalTok{E <-}\StringTok{ }\KeywordTok{list}\NormalTok{(e1,e2,e3,e4,e5,e6,e7,e8,e9,e10)}
\NormalTok{DT <-}\StringTok{ }\NormalTok{DecisionTree}\OperatorTok{$}\KeywordTok{new}\NormalTok{(V,E)}
\end{Highlighting}
\end{Shaded}

In the manufacturer's model, the uncertainties in the probabilities
associated with the polytomous chance nodes were modelled as independent
variables. This is not recommended because there is a chance that a
particular run of the PSA will yield probabilities that are outside the
range {[}0,1{]}. Representing the uncertain probabilities with draws
from a Dirichlet distribution is preferred. Creating a
\texttt{ChanceNode} with ModVars is permitted, but results in a warning
being issued.

\hypertarget{summary-of-the-model}{%
\subsection{Summary of the model}\label{summary-of-the-model}}

The model variables and their operands associated with a node and
(optionally) its descendants can be tabulated using the method
\texttt{tabulate\_modvars}. This returns a data frame describing each
variable, its description, units and uncertainty distribution. Variables
inheriting from type \texttt{ModVar} will be included in the tabulation;
regular numeric values will not be listed. For extensive models,
variables associated with separate branches of a tree can be tabulated
separately by calling the method for different head nodes.

The operands of model variables which are expressions of other model
variables can be included in the tabulation via the
\texttt{include.operands} parameter. This is recursive, allowing the
complete structure of a model, \emph{i.e.} its model variables and the
way in which they are combined, to be tabulated. In the Tegaderm model,
the complete structure is as follows:

\begin{longtable}[]{@{}ll@{}}
\toprule
Description & Distribution\tabularnewline
\midrule
\endhead
Dermatitis cost & Ga(30,5)\tabularnewline
P(dermatitis\textbar standard dressing) & n.dressings *
r.Dermatitis\tabularnewline
No.~dressings & N(3,0.3)\tabularnewline
Baseline dermatitis risk & N(0.0026,0.00026)\tabularnewline
LSI cost & Ga(50,5)\tabularnewline
Baseline LSI rate & N(0.1,0.01)\tabularnewline
CRBSI cost & Ga(198,50)\tabularnewline
P(CRBSI\textbar standard dressing) & r.CRBSI *
n.cathdays/1000\tabularnewline
Baseline CRBSI rate & N(1.48,0.074)\tabularnewline
No.~days with catheter & N(10,2)\tabularnewline
P(No comp\textbar standard dressing) & 1 - (p.Dermatitis.S + r.LSI +
p.CRBSI.S)\tabularnewline
P(dermatitis\textbar Tegaderm) & n.dressings * r.Dermatitis *
rr.Dermatitis\tabularnewline
Tegaderm Dermatitis RR & LN1(1.482,0.49)\tabularnewline
P(LSI\textbar Tegaderm) & r.LSI * hr.LSI\tabularnewline
Tegaderm LSI HR & LN1(-0.911,0.393)\tabularnewline
P(CRBSI\textbar Tegaderm) & r.CRBSI * n.cathdays *
hr.CRBSI/1000\tabularnewline
Tegaderm CRBSI HR & LN1(-0.911,0.393)\tabularnewline
P(No comp\textbar Tegaderm) & 1 - (p.Dermatitis.T + r.LSI.T +
p.CRBSI.T)\tabularnewline
Standard dressing cost & Const(1.34)\tabularnewline
Tegaderm CHG cost & Const(6.21)\tabularnewline
\bottomrule
\end{longtable}

\hypertarget{point-estimates-and-distributions-of-model-variables}{%
\subsection{Point estimates and distributions of model
variables}\label{point-estimates-and-distributions-of-model-variables}}

The point estimates, units and distributional properties are obtained
from the same call, in the remaining columns. Rows with \texttt{Qhat}
indicate that the quantiles have been estimated from simulation.

\begin{longtable}[]{@{}llrlll@{}}
\toprule
Description & Units & Mean & Q2.5 & Q97.5 & Qhat\tabularnewline
\midrule
\endhead
Dermatitis cost & GBP & 150.000 & NA & NA & NA\tabularnewline
P(dermatitis\textbar standard dressing) & P & 0.008 & NA & NA &
NA\tabularnewline
No.~dressings & dressings & 3.000 & NA & NA & NA\tabularnewline
Baseline dermatitis risk & /catheter & 0.003 & NA & NA &
NA\tabularnewline
LSI cost & GBP & 250.000 & NA & NA & NA\tabularnewline
Baseline LSI rate & /patient & 0.100 & NA & NA & NA\tabularnewline
CRBSI cost & GBP & 9900.000 & NA & NA & NA\tabularnewline
P(CRBSI\textbar standard dressing) & P & 0.015 & NA & NA &
NA\tabularnewline
Baseline CRBSI rate & /1000 catheter days & 1.480 & NA & NA &
NA\tabularnewline
No.~days with catheter & days & 10.000 & NA & NA & NA\tabularnewline
P(No comp\textbar standard dressing) & P & 0.877 & NA & NA &
NA\tabularnewline
P(dermatitis\textbar Tegaderm) & P & 0.039 & NA & NA & NA\tabularnewline
Tegaderm Dermatitis RR & ratio & 4.963 & NA & NA & NA\tabularnewline
P(LSI\textbar Tegaderm) & P & 0.043 & NA & NA & NA\tabularnewline
Tegaderm LSI HR & ratio & 0.434 & NA & NA & NA\tabularnewline
P(CRBSI\textbar Tegaderm) & P & 0.006 & NA & NA & NA\tabularnewline
Tegaderm CRBSI HR & ratio & 0.434 & NA & NA & NA\tabularnewline
P(No comp\textbar Tegaderm) & P & 0.911 & NA & NA & NA\tabularnewline
Standard dressing cost & GBP & 1.340 & NA & NA & NA\tabularnewline
Tegaderm CHG cost & GBP & 6.210 & NA & NA & NA\tabularnewline
\bottomrule
\end{longtable}

\hypertarget{running-the-model}{%
\subsection{Running the model}\label{running-the-model}}

The following code runs a single model scenario, using the
\texttt{evaluatePathways} method of a decision node to evaluate each
pathway from the decision node. In the model there are eight possible
root-to-leaf paths, each of which begins with the decision node and ends
with a leaf node. For example, pathway
\texttt{Dermatitis\ (Standard\ Dressing)} involves a traversal of nodes
\texttt{d}, \texttt{chance.S}, and \texttt{leaf.S.Dermatitis}. The
method \texttt{evaluateChoices} is similar, but aggregates the results
by choice. The results of the scenario model, using the code from the
previous section, yields the following table. This model did not
consider utility, and the columns associated with utility are removed.

\hypertarget{references}{%
\section*{References}\label{references}}
\addcontentsline{toc}{section}{References}

\hypertarget{refs}{}
\leavevmode\hypertarget{ref-jenks:2016a}{}%
1 Jenks M, Craig JA, Green W \emph{et al.} Tegaderm CHG IV securement
dressing for central venous and arterial catheter insertion sites: A
NICE Medical Technology Guidance. \emph{Applied Health Economics and
Health Policy} 2016;\textbf{14}:135--49.

\leavevmode\hypertarget{ref-briggs:2002a}{}%
2 Briggs A, Claxton K, Sculpher M. \emph{Decision modelling for health
economic evaluation}. Oxford, UK:: Oxford University Press 2006.

\end{document}
