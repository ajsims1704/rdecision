\PassOptionsToPackage{unicode=true}{hyperref} % options for packages loaded elsewhere
\PassOptionsToPackage{hyphens}{url}
%
\documentclass[]{article}
\usepackage{lmodern}
\usepackage{amssymb,amsmath}
\usepackage{ifxetex,ifluatex}
\usepackage{fixltx2e} % provides \textsubscript
\ifnum 0\ifxetex 1\fi\ifluatex 1\fi=0 % if pdftex
  \usepackage[T1]{fontenc}
  \usepackage[utf8]{inputenc}
  \usepackage{textcomp} % provides euro and other symbols
\else % if luatex or xelatex
  \usepackage{unicode-math}
  \defaultfontfeatures{Ligatures=TeX,Scale=MatchLowercase}
\fi
% use upquote if available, for straight quotes in verbatim environments
\IfFileExists{upquote.sty}{\usepackage{upquote}}{}
% use microtype if available
\IfFileExists{microtype.sty}{%
\usepackage[]{microtype}
\UseMicrotypeSet[protrusion]{basicmath} % disable protrusion for tt fonts
}{}
\IfFileExists{parskip.sty}{%
\usepackage{parskip}
}{% else
\setlength{\parindent}{0pt}
\setlength{\parskip}{6pt plus 2pt minus 1pt}
}
\usepackage{hyperref}
\hypersetup{
            pdftitle={Sumatriptan versus caffeine for migraine},
            pdfauthor={Andrew J. Sims},
            pdfborder={0 0 0},
            breaklinks=true}
\urlstyle{same}  % don't use monospace font for urls
\usepackage[margin=1in]{geometry}
\usepackage{color}
\usepackage{fancyvrb}
\newcommand{\VerbBar}{|}
\newcommand{\VERB}{\Verb[commandchars=\\\{\}]}
\DefineVerbatimEnvironment{Highlighting}{Verbatim}{commandchars=\\\{\}}
% Add ',fontsize=\small' for more characters per line
\usepackage{framed}
\definecolor{shadecolor}{RGB}{248,248,248}
\newenvironment{Shaded}{\begin{snugshade}}{\end{snugshade}}
\newcommand{\AlertTok}[1]{\textcolor[rgb]{0.94,0.16,0.16}{#1}}
\newcommand{\AnnotationTok}[1]{\textcolor[rgb]{0.56,0.35,0.01}{\textbf{\textit{#1}}}}
\newcommand{\AttributeTok}[1]{\textcolor[rgb]{0.77,0.63,0.00}{#1}}
\newcommand{\BaseNTok}[1]{\textcolor[rgb]{0.00,0.00,0.81}{#1}}
\newcommand{\BuiltInTok}[1]{#1}
\newcommand{\CharTok}[1]{\textcolor[rgb]{0.31,0.60,0.02}{#1}}
\newcommand{\CommentTok}[1]{\textcolor[rgb]{0.56,0.35,0.01}{\textit{#1}}}
\newcommand{\CommentVarTok}[1]{\textcolor[rgb]{0.56,0.35,0.01}{\textbf{\textit{#1}}}}
\newcommand{\ConstantTok}[1]{\textcolor[rgb]{0.00,0.00,0.00}{#1}}
\newcommand{\ControlFlowTok}[1]{\textcolor[rgb]{0.13,0.29,0.53}{\textbf{#1}}}
\newcommand{\DataTypeTok}[1]{\textcolor[rgb]{0.13,0.29,0.53}{#1}}
\newcommand{\DecValTok}[1]{\textcolor[rgb]{0.00,0.00,0.81}{#1}}
\newcommand{\DocumentationTok}[1]{\textcolor[rgb]{0.56,0.35,0.01}{\textbf{\textit{#1}}}}
\newcommand{\ErrorTok}[1]{\textcolor[rgb]{0.64,0.00,0.00}{\textbf{#1}}}
\newcommand{\ExtensionTok}[1]{#1}
\newcommand{\FloatTok}[1]{\textcolor[rgb]{0.00,0.00,0.81}{#1}}
\newcommand{\FunctionTok}[1]{\textcolor[rgb]{0.00,0.00,0.00}{#1}}
\newcommand{\ImportTok}[1]{#1}
\newcommand{\InformationTok}[1]{\textcolor[rgb]{0.56,0.35,0.01}{\textbf{\textit{#1}}}}
\newcommand{\KeywordTok}[1]{\textcolor[rgb]{0.13,0.29,0.53}{\textbf{#1}}}
\newcommand{\NormalTok}[1]{#1}
\newcommand{\OperatorTok}[1]{\textcolor[rgb]{0.81,0.36,0.00}{\textbf{#1}}}
\newcommand{\OtherTok}[1]{\textcolor[rgb]{0.56,0.35,0.01}{#1}}
\newcommand{\PreprocessorTok}[1]{\textcolor[rgb]{0.56,0.35,0.01}{\textit{#1}}}
\newcommand{\RegionMarkerTok}[1]{#1}
\newcommand{\SpecialCharTok}[1]{\textcolor[rgb]{0.00,0.00,0.00}{#1}}
\newcommand{\SpecialStringTok}[1]{\textcolor[rgb]{0.31,0.60,0.02}{#1}}
\newcommand{\StringTok}[1]{\textcolor[rgb]{0.31,0.60,0.02}{#1}}
\newcommand{\VariableTok}[1]{\textcolor[rgb]{0.00,0.00,0.00}{#1}}
\newcommand{\VerbatimStringTok}[1]{\textcolor[rgb]{0.31,0.60,0.02}{#1}}
\newcommand{\WarningTok}[1]{\textcolor[rgb]{0.56,0.35,0.01}{\textbf{\textit{#1}}}}
\usepackage{longtable,booktabs}
% Fix footnotes in tables (requires footnote package)
\IfFileExists{footnote.sty}{\usepackage{footnote}\makesavenoteenv{longtable}}{}
\usepackage{graphicx,grffile}
\makeatletter
\def\maxwidth{\ifdim\Gin@nat@width>\linewidth\linewidth\else\Gin@nat@width\fi}
\def\maxheight{\ifdim\Gin@nat@height>\textheight\textheight\else\Gin@nat@height\fi}
\makeatother
% Scale images if necessary, so that they will not overflow the page
% margins by default, and it is still possible to overwrite the defaults
% using explicit options in \includegraphics[width, height, ...]{}
\setkeys{Gin}{width=\maxwidth,height=\maxheight,keepaspectratio}
\setlength{\emergencystretch}{3em}  % prevent overfull lines
\providecommand{\tightlist}{%
  \setlength{\itemsep}{0pt}\setlength{\parskip}{0pt}}
\setcounter{secnumdepth}{0}
% Redefines (sub)paragraphs to behave more like sections
\ifx\paragraph\undefined\else
\let\oldparagraph\paragraph
\renewcommand{\paragraph}[1]{\oldparagraph{#1}\mbox{}}
\fi
\ifx\subparagraph\undefined\else
\let\oldsubparagraph\subparagraph
\renewcommand{\subparagraph}[1]{\oldsubparagraph{#1}\mbox{}}
\fi

% set default figure placement to htbp
\makeatletter
\def\fps@figure{htbp}
\makeatother

\usepackage{etoolbox}
\makeatletter
\providecommand{\subtitle}[1]{% add subtitle to \maketitle
  \apptocmd{\@title}{\par {\large #1 \par}}{}{}
}
\makeatother

\title{Sumatriptan versus caffeine for migraine}
\providecommand{\subtitle}[1]{}
\subtitle{A decision tree example}
\author{Andrew J. Sims}
\date{2020-05-30}

\begin{document}
\maketitle

\hypertarget{introduction}{%
\section{Introduction}\label{introduction}}

This vignette is an example of modelling a decision tree using the
\texttt{rdecision} package. It is based on the example given by
Briggs\textsuperscript{1} which itself is based on a decision tree which
compared oral Sumatriptan versus oral caffeine/Ergotamine for
migraine\textsuperscript{2}.

\hypertarget{creating-the-model}{%
\section{Creating the model}\label{creating-the-model}}

The following code constructs the decision tree, node by node. In the
formulation used by \texttt{rdecision}, each node is a potentially
recursive structure which is allowed to have zero or more child nodes;
any child nodes must have already been declared before their parent node
is declared. This implies that a tree should be constructed from right
to left, starting with leaf nodes which have no children (leaf nodes are
synonymous with pathways in Briggs' terminology, and called `States' in
\texttt{rdecision}). The final node to be constructed is the left-most
decision node in the model.

\begin{Shaded}
\begin{Highlighting}[]
\CommentTok{# Time horizon}
\NormalTok{th <-}\StringTok{ }\KeywordTok{as.difftime}\NormalTok{(}\DecValTok{48}\NormalTok{, }\DataTypeTok{units=}\StringTok{"hours"}\NormalTok{)}

\CommentTok{# Model variables}
\NormalTok{sumatriptan <-}\StringTok{ }\FloatTok{16.10}
\NormalTok{caffeine <-}\StringTok{ }\FloatTok{1.32}
\NormalTok{ED <-}\StringTok{ }\FloatTok{63.16}
\NormalTok{admission <-}\StringTok{ }\DecValTok{1093}

\CommentTok{# Sumatriptan branch}
\NormalTok{state.a <-}\StringTok{ }\NormalTok{State}\OperatorTok{$}\KeywordTok{new}\NormalTok{(}\StringTok{"A"}\NormalTok{, }\DataTypeTok{cost=}\NormalTok{(sumatriptan), }\DataTypeTok{utility=}\FloatTok{1.0}\NormalTok{, }\DataTypeTok{interval=}\NormalTok{th)}
\NormalTok{state.b <-}\StringTok{ }\NormalTok{State}\OperatorTok{$}\KeywordTok{new}\NormalTok{(}\StringTok{"B"}\NormalTok{, }\DataTypeTok{cost=}\NormalTok{(}\DecValTok{2}\OperatorTok{*}\NormalTok{sumatriptan), }\DataTypeTok{utility=}\FloatTok{0.9}\NormalTok{, }\DataTypeTok{interval=}\NormalTok{th)}
\NormalTok{state.c <-}\StringTok{ }\NormalTok{State}\OperatorTok{$}\KeywordTok{new}\NormalTok{(}\StringTok{"C"}\NormalTok{, }\DataTypeTok{cost=}\NormalTok{(sumatriptan), }\DataTypeTok{utility=}\OperatorTok{-}\FloatTok{0.3}\NormalTok{, }\DataTypeTok{interval=}\NormalTok{th)}
\NormalTok{state.d <-}\StringTok{ }\NormalTok{State}\OperatorTok{$}\KeywordTok{new}\NormalTok{(}\StringTok{"D"}\NormalTok{, }\DataTypeTok{cost=}\NormalTok{(sumatriptan}\OperatorTok{+}\NormalTok{ED), }\DataTypeTok{utility=}\FloatTok{0.1}\NormalTok{, }\DataTypeTok{interval=}\NormalTok{th)}
\NormalTok{state.e <-}\StringTok{ }\NormalTok{State}\OperatorTok{$}\KeywordTok{new}\NormalTok{(}\StringTok{"E"}\NormalTok{, }\DataTypeTok{cost=}\NormalTok{(sumatriptan}\OperatorTok{+}\NormalTok{ED}\OperatorTok{+}\NormalTok{admission), }\DataTypeTok{utility=}\OperatorTok{-}\FloatTok{0.3}\NormalTok{, }\DataTypeTok{interval=}\NormalTok{th)}

\NormalTok{c}\FloatTok{.8}\NormalTok{ <-}\StringTok{ }\NormalTok{ChanceNode}\OperatorTok{$}\KeywordTok{new}\NormalTok{(}
  \DataTypeTok{p =} \KeywordTok{list}\NormalTok{(}\FloatTok{0.998}\NormalTok{, }\FloatTok{0.002}\NormalTok{),}
  \DataTypeTok{children =} \KeywordTok{list}\NormalTok{(state.d, state.e),}
  \DataTypeTok{edgelabels =} \KeywordTok{list}\NormalTok{(}\StringTok{"Relief"}\NormalTok{, }\StringTok{"Hospitalization"}\NormalTok{),}
  \DataTypeTok{costs =} \KeywordTok{list}\NormalTok{(}\DecValTok{0}\NormalTok{, }\DecValTok{0}\NormalTok{)}
\NormalTok{)}

\NormalTok{c}\FloatTok{.4}\NormalTok{ <-}\StringTok{ }\NormalTok{ChanceNode}\OperatorTok{$}\KeywordTok{new}\NormalTok{(}
  \DataTypeTok{p =} \KeywordTok{list}\NormalTok{(}\FloatTok{0.594}\NormalTok{, }\FloatTok{0.406}\NormalTok{),}
  \DataTypeTok{children =} \KeywordTok{list}\NormalTok{(state.a, state.b),}
  \DataTypeTok{edgelabels =} \KeywordTok{list}\NormalTok{(}\StringTok{"No recurrence"}\NormalTok{, }\StringTok{"Recurrence relieved with 2nd dose"}\NormalTok{),}
  \DataTypeTok{costs =} \KeywordTok{list}\NormalTok{(}\DecValTok{0}\NormalTok{, }\DecValTok{0}\NormalTok{)}
\NormalTok{)}

\NormalTok{c}\FloatTok{.5}\NormalTok{ <-}\StringTok{ }\NormalTok{ChanceNode}\OperatorTok{$}\KeywordTok{new}\NormalTok{(}
  \DataTypeTok{p =} \KeywordTok{list}\NormalTok{(}\FloatTok{0.920}\NormalTok{, }\FloatTok{0.080}\NormalTok{),}
  \DataTypeTok{children =} \KeywordTok{list}\NormalTok{(state.c, c}\FloatTok{.8}\NormalTok{),}
  \DataTypeTok{edgelabels =} \KeywordTok{list}\NormalTok{(}\StringTok{"Endures attack"}\NormalTok{, }\StringTok{"ER"}\NormalTok{),}
  \DataTypeTok{costs =} \KeywordTok{list}\NormalTok{(}\DecValTok{0}\NormalTok{, }\DecValTok{0}\NormalTok{)}
\NormalTok{)}

\NormalTok{c}\FloatTok{.2}\NormalTok{ <-}\StringTok{ }\NormalTok{ChanceNode}\OperatorTok{$}\KeywordTok{new}\NormalTok{(}
  \DataTypeTok{p =} \KeywordTok{list}\NormalTok{(}\FloatTok{0.558}\NormalTok{, }\FloatTok{0.442}\NormalTok{),}
  \DataTypeTok{children =} \KeywordTok{list}\NormalTok{(c}\FloatTok{.4}\NormalTok{, c}\FloatTok{.5}\NormalTok{),}
  \DataTypeTok{edgelabels =} \KeywordTok{list}\NormalTok{(}\StringTok{"Relief"}\NormalTok{, }\StringTok{"No relief"}\NormalTok{),}
  \DataTypeTok{costs =} \KeywordTok{list}\NormalTok{(}\DecValTok{0}\NormalTok{, }\DecValTok{0}\NormalTok{)}
\NormalTok{)}

\CommentTok{# Caffeine/Ergotamine branch}
\NormalTok{state.f <-}\StringTok{ }\NormalTok{State}\OperatorTok{$}\KeywordTok{new}\NormalTok{(}\StringTok{"F"}\NormalTok{, }\DataTypeTok{cost=}\NormalTok{(caffeine), }\DataTypeTok{utility=}\FloatTok{1.0}\NormalTok{, }\DataTypeTok{interval=}\NormalTok{th)}
\NormalTok{state.g <-}\StringTok{ }\NormalTok{State}\OperatorTok{$}\KeywordTok{new}\NormalTok{(}\StringTok{"G"}\NormalTok{, }\DataTypeTok{cost=}\NormalTok{(}\DecValTok{2}\OperatorTok{*}\NormalTok{caffeine), }\DataTypeTok{utility=}\FloatTok{0.9}\NormalTok{, }\DataTypeTok{interval=}\NormalTok{th)}
\NormalTok{state.h <-}\StringTok{ }\NormalTok{State}\OperatorTok{$}\KeywordTok{new}\NormalTok{(}\StringTok{"H"}\NormalTok{, }\DataTypeTok{cost=}\NormalTok{(caffeine), }\DataTypeTok{utility=}\OperatorTok{-}\FloatTok{0.3}\NormalTok{, }\DataTypeTok{interval=}\NormalTok{th)}
\NormalTok{state.i <-}\StringTok{ }\NormalTok{State}\OperatorTok{$}\KeywordTok{new}\NormalTok{(}\StringTok{"I"}\NormalTok{, }\DataTypeTok{cost=}\NormalTok{(caffeine}\OperatorTok{+}\NormalTok{ED), }\DataTypeTok{utility=}\FloatTok{0.1}\NormalTok{, }\DataTypeTok{interval=}\NormalTok{th)}
\NormalTok{state.j <-}\StringTok{ }\NormalTok{State}\OperatorTok{$}\KeywordTok{new}\NormalTok{(}\StringTok{"J"}\NormalTok{, }\DataTypeTok{cost=}\NormalTok{(caffeine}\OperatorTok{+}\NormalTok{ED}\OperatorTok{+}\NormalTok{admission), }\DataTypeTok{utility=}\OperatorTok{-}\FloatTok{0.3}\NormalTok{, }\DataTypeTok{interval=}\NormalTok{th)}

\NormalTok{c}\FloatTok{.9}\NormalTok{ <-}\StringTok{ }\NormalTok{ChanceNode}\OperatorTok{$}\KeywordTok{new}\NormalTok{(}
  \DataTypeTok{p =} \KeywordTok{list}\NormalTok{(}\FloatTok{0.998}\NormalTok{, }\FloatTok{0.002}\NormalTok{),}
  \DataTypeTok{children =} \KeywordTok{list}\NormalTok{(state.i, state.j),}
  \DataTypeTok{edgelabels =} \KeywordTok{list}\NormalTok{(}\StringTok{"Relief"}\NormalTok{, }\StringTok{"Hospitalization"}\NormalTok{),}
  \DataTypeTok{costs =} \KeywordTok{list}\NormalTok{(}\DecValTok{0}\NormalTok{, }\DecValTok{0}\NormalTok{)}
\NormalTok{)}

\NormalTok{c}\FloatTok{.6}\NormalTok{ <-}\StringTok{ }\NormalTok{ChanceNode}\OperatorTok{$}\KeywordTok{new}\NormalTok{(}
  \DataTypeTok{p =} \KeywordTok{list}\NormalTok{(}\FloatTok{0.703}\NormalTok{, }\FloatTok{0.297}\NormalTok{),}
  \DataTypeTok{children =} \KeywordTok{list}\NormalTok{(state.f, state.g),}
  \DataTypeTok{edgelabels =} \KeywordTok{list}\NormalTok{(}\StringTok{"No recurrence"}\NormalTok{, }\StringTok{"Recurrence relieved with 2nd dose"}\NormalTok{),}
  \DataTypeTok{costs =} \KeywordTok{list}\NormalTok{(}\DecValTok{0}\NormalTok{, }\DecValTok{0}\NormalTok{)}
\NormalTok{)}

\NormalTok{c}\FloatTok{.7}\NormalTok{ <-}\StringTok{ }\NormalTok{ChanceNode}\OperatorTok{$}\KeywordTok{new}\NormalTok{(}
  \DataTypeTok{p =} \KeywordTok{list}\NormalTok{(}\FloatTok{0.920}\NormalTok{, }\FloatTok{0.080}\NormalTok{),}
  \DataTypeTok{children =} \KeywordTok{list}\NormalTok{(state.h, c}\FloatTok{.9}\NormalTok{),}
  \DataTypeTok{edgelabels =} \KeywordTok{list}\NormalTok{(}\StringTok{"Endures attack"}\NormalTok{, }\StringTok{"ER"}\NormalTok{),}
  \DataTypeTok{costs =} \KeywordTok{list}\NormalTok{(}\DecValTok{0}\NormalTok{, }\DecValTok{0}\NormalTok{)}
\NormalTok{)}

\NormalTok{c}\FloatTok{.3}\NormalTok{ <-}\StringTok{ }\NormalTok{ChanceNode}\OperatorTok{$}\KeywordTok{new}\NormalTok{(}
  \DataTypeTok{p =} \KeywordTok{list}\NormalTok{(}\FloatTok{0.379}\NormalTok{, }\FloatTok{0.621}\NormalTok{),}
  \DataTypeTok{children =} \KeywordTok{list}\NormalTok{(c}\FloatTok{.6}\NormalTok{, c}\FloatTok{.7}\NormalTok{),}
  \DataTypeTok{edgelabels =} \KeywordTok{list}\NormalTok{(}\StringTok{"Relief"}\NormalTok{, }\StringTok{"No relief"}\NormalTok{),}
  \DataTypeTok{costs =} \KeywordTok{list}\NormalTok{(}\DecValTok{0}\NormalTok{, }\DecValTok{0}\NormalTok{)}
\NormalTok{)}

\CommentTok{# decision node}
\NormalTok{d}\FloatTok{.1}\NormalTok{ <-}\StringTok{ }\NormalTok{DecisionNode}\OperatorTok{$}\KeywordTok{new}\NormalTok{(}
  \DataTypeTok{children =} \KeywordTok{list}\NormalTok{(c}\FloatTok{.2}\NormalTok{, c}\FloatTok{.3}\NormalTok{),}
  \DataTypeTok{edgelabels =} \KeywordTok{list}\NormalTok{(}\StringTok{"Sumatriptan"}\NormalTok{, }\StringTok{"Caffeine/Ergotamine"}\NormalTok{),}
  \DataTypeTok{costs =} \KeywordTok{list}\NormalTok{(}\DecValTok{0}\NormalTok{, }\DecValTok{0}\NormalTok{)}
\NormalTok{)}
\end{Highlighting}
\end{Shaded}

\hypertarget{running-the-model}{%
\section{Running the model}\label{running-the-model}}

The method \texttt{evaluatePathways} of decision nodes computes the
probability, cost and utility of traversing each root-to-leaf path in
the model. In the Sumatriptan model there are eight such paths, each of
which begins with the decision node and ends with a leaf node. For
example, pathway A involves a traversal of nodes \texttt{d.1},
\texttt{c.2}, \texttt{c.4} and \texttt{state.a}.

\hypertarget{model-results}{%
\section{Model results}\label{model-results}}

The results of the scenario model, using the code from the previous
sections, yields the following result:

\begin{longtable}[]{@{}llrrrrr@{}}
\toprule
Choice & Pathway & Probability & Cost & Expected Cost & Utility &
Expected Utility\tabularnewline
\midrule
\endhead
Sumatriptan & A & 0.331 & 16.10 & 5.34 & 1.0 & 0.33145\tabularnewline
Sumatriptan & B & 0.227 & 32.20 & 7.29 & 0.9 & 0.20389\tabularnewline
Sumatriptan & C & 0.407 & 16.10 & 6.55 & -0.3 & -0.12199\tabularnewline
Sumatriptan & D & 0.035 & 79.26 & 2.80 & 0.1 & 0.00353\tabularnewline
Sumatriptan & E & 0.000 & 1172.26 & 0.08 & -0.3 &
-0.00002\tabularnewline
Caffeine/Ergotamine & F & 0.266 & 1.32 & 0.35 & 1.0 &
0.26644\tabularnewline
Caffeine/Ergotamine & G & 0.113 & 2.64 & 0.30 & 0.9 &
0.10131\tabularnewline
Caffeine/Ergotamine & H & 0.571 & 1.32 & 0.75 & -0.3 &
-0.17140\tabularnewline
Caffeine/Ergotamine & I & 0.050 & 64.48 & 3.20 & 0.1 &
0.00496\tabularnewline
Caffeine/Ergotamine & J & 0.000 & 1157.48 & 0.12 & -0.3 &
-0.00003\tabularnewline
\bottomrule
\end{longtable}

There are, as expected, eight root-to-leaf pathways. The total
probability, expected cost and expected utility for each choice can be
calculated from the table above, or by invoking the
\texttt{evaluateChoices} method of a decision node. This gives the
following result, consistent with that reported by Evans \emph{et
al}\textsuperscript{2}.

\begin{longtable}[]{@{}lrr@{}}
\toprule
Choice & Expected Cost & Expected Utility\tabularnewline
\midrule
\endhead
Caffeine/Ergotamine & 4.71 & 0.20128\tabularnewline
Sumatriptan & 22.06 & 0.41686\tabularnewline
\bottomrule
\end{longtable}

\hypertarget{references}{%
\section*{References}\label{references}}
\addcontentsline{toc}{section}{References}

\hypertarget{refs}{}
\leavevmode\hypertarget{ref-briggs:2002a}{}%
1. Briggs, A., Claxton, K. \& Sculpher, M. \emph{Decision modelling for
health economic evaluation}. (Oxford University Press, 2006).

\leavevmode\hypertarget{ref-evans:1997a}{}%
2. Evans, K. W., Boan, J. A., Evans, J. L. \& Shuaib, A. Economic
evaluation of oral sumatriptan compared with oral caffeine/ergotamine
for migraine. \emph{Pharmacoeconomics} \textbf{12}, 565--577 (1997).

\end{document}
