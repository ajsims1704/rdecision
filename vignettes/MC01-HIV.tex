% Options for packages loaded elsewhere
\PassOptionsToPackage{unicode}{hyperref}
\PassOptionsToPackage{hyphens}{url}
%
\documentclass[
]{article}
\usepackage{amsmath,amssymb}
\usepackage{lmodern}
\usepackage{ifxetex,ifluatex}
\ifnum 0\ifxetex 1\fi\ifluatex 1\fi=0 % if pdftex
  \usepackage[T1]{fontenc}
  \usepackage[utf8]{inputenc}
  \usepackage{textcomp} % provide euro and other symbols
\else % if luatex or xetex
  \usepackage{unicode-math}
  \defaultfontfeatures{Scale=MatchLowercase}
  \defaultfontfeatures[\rmfamily]{Ligatures=TeX,Scale=1}
\fi
% Use upquote if available, for straight quotes in verbatim environments
\IfFileExists{upquote.sty}{\usepackage{upquote}}{}
\IfFileExists{microtype.sty}{% use microtype if available
  \usepackage[]{microtype}
  \UseMicrotypeSet[protrusion]{basicmath} % disable protrusion for tt fonts
}{}
\makeatletter
\@ifundefined{KOMAClassName}{% if non-KOMA class
  \IfFileExists{parskip.sty}{%
    \usepackage{parskip}
  }{% else
    \setlength{\parindent}{0pt}
    \setlength{\parskip}{6pt plus 2pt minus 1pt}}
}{% if KOMA class
  \KOMAoptions{parskip=half}}
\makeatother
\usepackage{xcolor}
\IfFileExists{xurl.sty}{\usepackage{xurl}}{} % add URL line breaks if available
\IfFileExists{bookmark.sty}{\usepackage{bookmark}}{\usepackage{hyperref}}
\hypersetup{
  pdftitle={Elementary Markov Model (Chancellor 1997)},
  pdfauthor={Andrew J. Sims},
  hidelinks,
  pdfcreator={LaTeX via pandoc}}
\urlstyle{same} % disable monospaced font for URLs
\usepackage[margin=1in]{geometry}
\usepackage{color}
\usepackage{fancyvrb}
\newcommand{\VerbBar}{|}
\newcommand{\VERB}{\Verb[commandchars=\\\{\}]}
\DefineVerbatimEnvironment{Highlighting}{Verbatim}{commandchars=\\\{\}}
% Add ',fontsize=\small' for more characters per line
\usepackage{framed}
\definecolor{shadecolor}{RGB}{248,248,248}
\newenvironment{Shaded}{\begin{snugshade}}{\end{snugshade}}
\newcommand{\AlertTok}[1]{\textcolor[rgb]{0.94,0.16,0.16}{#1}}
\newcommand{\AnnotationTok}[1]{\textcolor[rgb]{0.56,0.35,0.01}{\textbf{\textit{#1}}}}
\newcommand{\AttributeTok}[1]{\textcolor[rgb]{0.77,0.63,0.00}{#1}}
\newcommand{\BaseNTok}[1]{\textcolor[rgb]{0.00,0.00,0.81}{#1}}
\newcommand{\BuiltInTok}[1]{#1}
\newcommand{\CharTok}[1]{\textcolor[rgb]{0.31,0.60,0.02}{#1}}
\newcommand{\CommentTok}[1]{\textcolor[rgb]{0.56,0.35,0.01}{\textit{#1}}}
\newcommand{\CommentVarTok}[1]{\textcolor[rgb]{0.56,0.35,0.01}{\textbf{\textit{#1}}}}
\newcommand{\ConstantTok}[1]{\textcolor[rgb]{0.00,0.00,0.00}{#1}}
\newcommand{\ControlFlowTok}[1]{\textcolor[rgb]{0.13,0.29,0.53}{\textbf{#1}}}
\newcommand{\DataTypeTok}[1]{\textcolor[rgb]{0.13,0.29,0.53}{#1}}
\newcommand{\DecValTok}[1]{\textcolor[rgb]{0.00,0.00,0.81}{#1}}
\newcommand{\DocumentationTok}[1]{\textcolor[rgb]{0.56,0.35,0.01}{\textbf{\textit{#1}}}}
\newcommand{\ErrorTok}[1]{\textcolor[rgb]{0.64,0.00,0.00}{\textbf{#1}}}
\newcommand{\ExtensionTok}[1]{#1}
\newcommand{\FloatTok}[1]{\textcolor[rgb]{0.00,0.00,0.81}{#1}}
\newcommand{\FunctionTok}[1]{\textcolor[rgb]{0.00,0.00,0.00}{#1}}
\newcommand{\ImportTok}[1]{#1}
\newcommand{\InformationTok}[1]{\textcolor[rgb]{0.56,0.35,0.01}{\textbf{\textit{#1}}}}
\newcommand{\KeywordTok}[1]{\textcolor[rgb]{0.13,0.29,0.53}{\textbf{#1}}}
\newcommand{\NormalTok}[1]{#1}
\newcommand{\OperatorTok}[1]{\textcolor[rgb]{0.81,0.36,0.00}{\textbf{#1}}}
\newcommand{\OtherTok}[1]{\textcolor[rgb]{0.56,0.35,0.01}{#1}}
\newcommand{\PreprocessorTok}[1]{\textcolor[rgb]{0.56,0.35,0.01}{\textit{#1}}}
\newcommand{\RegionMarkerTok}[1]{#1}
\newcommand{\SpecialCharTok}[1]{\textcolor[rgb]{0.00,0.00,0.00}{#1}}
\newcommand{\SpecialStringTok}[1]{\textcolor[rgb]{0.31,0.60,0.02}{#1}}
\newcommand{\StringTok}[1]{\textcolor[rgb]{0.31,0.60,0.02}{#1}}
\newcommand{\VariableTok}[1]{\textcolor[rgb]{0.00,0.00,0.00}{#1}}
\newcommand{\VerbatimStringTok}[1]{\textcolor[rgb]{0.31,0.60,0.02}{#1}}
\newcommand{\WarningTok}[1]{\textcolor[rgb]{0.56,0.35,0.01}{\textbf{\textit{#1}}}}
\usepackage{graphicx}
\makeatletter
\def\maxwidth{\ifdim\Gin@nat@width>\linewidth\linewidth\else\Gin@nat@width\fi}
\def\maxheight{\ifdim\Gin@nat@height>\textheight\textheight\else\Gin@nat@height\fi}
\makeatother
% Scale images if necessary, so that they will not overflow the page
% margins by default, and it is still possible to overwrite the defaults
% using explicit options in \includegraphics[width, height, ...]{}
\setkeys{Gin}{width=\maxwidth,height=\maxheight,keepaspectratio}
% Set default figure placement to htbp
\makeatletter
\def\fps@figure{htbp}
\makeatother
\setlength{\emergencystretch}{3em} % prevent overfull lines
\providecommand{\tightlist}{%
  \setlength{\itemsep}{0pt}\setlength{\parskip}{0pt}}
\setcounter{secnumdepth}{-\maxdimen} % remove section numbering
\ifluatex
  \usepackage{selnolig}  % disable illegal ligatures
\fi
\newlength{\cslhangindent}
\setlength{\cslhangindent}{1.5em}
\newlength{\csllabelwidth}
\setlength{\csllabelwidth}{3em}
\newenvironment{CSLReferences}[2] % #1 hanging-ident, #2 entry spacing
 {% don't indent paragraphs
  \setlength{\parindent}{0pt}
  % turn on hanging indent if param 1 is 1
  \ifodd #1 \everypar{\setlength{\hangindent}{\cslhangindent}}\ignorespaces\fi
  % set entry spacing
  \ifnum #2 > 0
  \setlength{\parskip}{#2\baselineskip}
  \fi
 }%
 {}
\usepackage{calc}
\newcommand{\CSLBlock}[1]{#1\hfill\break}
\newcommand{\CSLLeftMargin}[1]{\parbox[t]{\csllabelwidth}{#1}}
\newcommand{\CSLRightInline}[1]{\parbox[t]{\linewidth - \csllabelwidth}{#1}\break}
\newcommand{\CSLIndent}[1]{\hspace{\cslhangindent}#1}

\title{Elementary Markov Model (Chancellor 1997)}
\usepackage{etoolbox}
\makeatletter
\providecommand{\subtitle}[1]{% add subtitle to \maketitle
  \apptocmd{\@title}{\par {\large #1 \par}}{}{}
}
\makeatother
\subtitle{Monotherapy versus combination therapy for HIV}
\author{Andrew J. Sims}
\date{May 2021}

\begin{document}
\maketitle

\hypertarget{introduction}{%
\section{Introduction}\label{introduction}}

This vignette is an example of an elementary cohort Markov model using
the \texttt{rdecision} package. It is based on the example given by
Briggs \emph{et al}\textsuperscript{1} (Exercise 2.5) which itself is
based on a Markov model described by Chancellor \emph{et
al}.\textsuperscript{2} The model compares a combination therapy of
Lamivudine/Zidovudine versus Zidovudine monotherapy in people with HIV
infection.

\hypertarget{creating-the-model}{%
\section{Creating the model}\label{creating-the-model}}

The variables used in the model are all numerical constants, and are
defined as follows. The original model was based on annual transition
probabilities; these are converted to instantaneous hazard rates in
units of events/year.

\begin{Shaded}
\begin{Highlighting}[]
\CommentTok{\# transition rates calculated from annual transition probabilities}
\NormalTok{trAB }\OtherTok{\textless{}{-}} \SpecialCharTok{{-}}\FunctionTok{log}\NormalTok{(}\DecValTok{1}\FloatTok{{-}0.202}\NormalTok{)}\SpecialCharTok{/}\DecValTok{1} 
\NormalTok{trAC }\OtherTok{\textless{}{-}} \SpecialCharTok{{-}}\FunctionTok{log}\NormalTok{(}\DecValTok{1}\FloatTok{{-}0.067}\NormalTok{)}\SpecialCharTok{/}\DecValTok{1}
\NormalTok{trAD }\OtherTok{\textless{}{-}} \SpecialCharTok{{-}}\FunctionTok{log}\NormalTok{(}\DecValTok{1}\FloatTok{{-}0.010}\NormalTok{)}\SpecialCharTok{/}\DecValTok{1}
\NormalTok{trBC }\OtherTok{\textless{}{-}} \SpecialCharTok{{-}}\FunctionTok{log}\NormalTok{(}\DecValTok{1}\FloatTok{{-}0.407}\NormalTok{)}\SpecialCharTok{/}\DecValTok{1}
\NormalTok{trBD }\OtherTok{\textless{}{-}} \SpecialCharTok{{-}}\FunctionTok{log}\NormalTok{(}\DecValTok{1}\FloatTok{{-}0.012}\NormalTok{)}\SpecialCharTok{/}\DecValTok{1}
\NormalTok{trCD }\OtherTok{\textless{}{-}} \SpecialCharTok{{-}}\FunctionTok{log}\NormalTok{(}\DecValTok{1}\FloatTok{{-}0.250}\NormalTok{)}\SpecialCharTok{/}\DecValTok{1}
\CommentTok{\# Costs}
\NormalTok{dmca }\OtherTok{\textless{}{-}} \DecValTok{1701} \CommentTok{\# direct medical costs associated with state A}
\NormalTok{dmcb }\OtherTok{\textless{}{-}} \DecValTok{1774} \CommentTok{\# direct medical costs associated with state B}
\NormalTok{dmcc }\OtherTok{\textless{}{-}} \DecValTok{6948} \CommentTok{\# direct medical costs associated with state C}
\NormalTok{ccca }\OtherTok{\textless{}{-}} \DecValTok{1055} \CommentTok{\# Community care costs associated with state A}
\NormalTok{cccb }\OtherTok{\textless{}{-}} \DecValTok{1278} \CommentTok{\# Community care costs associated with state B}
\NormalTok{cccc }\OtherTok{\textless{}{-}} \DecValTok{2059} \CommentTok{\# Community care costs associated with state C}
\CommentTok{\# Drug costs}
\NormalTok{cAZT }\OtherTok{\textless{}{-}} \DecValTok{2278} \CommentTok{\# zidovudine drug cost}
\NormalTok{cLam }\OtherTok{\textless{}{-}} \DecValTok{2086} \CommentTok{\# lamivudine drug cost}
\CommentTok{\# Other parameters}
\NormalTok{RR }\OtherTok{\textless{}{-}} \FloatTok{0.509} \CommentTok{\# treatment effect}
\NormalTok{cDR }\OtherTok{\textless{}{-}} \DecValTok{6} \CommentTok{\# annual discount rate, costs (\%)}
\NormalTok{oDR }\OtherTok{\textless{}{-}} \DecValTok{6} \CommentTok{\# annual discount rate, benefits (\%)}
\end{Highlighting}
\end{Shaded}

The monotherapy model is constructed by forming a graph, with each state
as a node and each transition as an edge. Nodes (of class
\texttt{MarkovState}) and edges (class \texttt{MarkovTransition}) have
various properties whose values reflect the variables of the model
(costs, rates etc.). The rate for one of the outgoing transitions from
each non-absorbing state is set to NULL to allow the sum of
probabilities leaving each state, per cycle, to be adjusted to 1. The
usual case, as here, is to set the self-loop rates to NULL (i.e.~the
probability of remaining in a state is given by one minus the
probability of leaving the state).

\begin{Shaded}
\begin{Highlighting}[]
\CommentTok{\# create Markov states for monotherapy (zidovudine only)}
\NormalTok{s.mono.A }\OtherTok{\textless{}{-}}\NormalTok{ MarkovState}\SpecialCharTok{$}\FunctionTok{new}\NormalTok{(}\StringTok{"A"}\NormalTok{, }\AttributeTok{cost=}\NormalTok{dmca}\SpecialCharTok{+}\NormalTok{ccca}\SpecialCharTok{+}\NormalTok{cAZT)}
\NormalTok{s.mono.B }\OtherTok{\textless{}{-}}\NormalTok{ MarkovState}\SpecialCharTok{$}\FunctionTok{new}\NormalTok{(}\StringTok{"B"}\NormalTok{, }\AttributeTok{cost=}\NormalTok{dmcb}\SpecialCharTok{+}\NormalTok{cccb}\SpecialCharTok{+}\NormalTok{cAZT)}
\NormalTok{s.mono.C }\OtherTok{\textless{}{-}}\NormalTok{ MarkovState}\SpecialCharTok{$}\FunctionTok{new}\NormalTok{(}\StringTok{"C"}\NormalTok{, }\AttributeTok{cost=}\NormalTok{dmcc}\SpecialCharTok{+}\NormalTok{cccc}\SpecialCharTok{+}\NormalTok{cAZT)}
\NormalTok{s.mono.D }\OtherTok{\textless{}{-}}\NormalTok{ MarkovState}\SpecialCharTok{$}\FunctionTok{new}\NormalTok{(}\StringTok{"D"}\NormalTok{, }\AttributeTok{cost=}\DecValTok{0}\NormalTok{)}
\CommentTok{\# create transitions}
\NormalTok{tAA }\OtherTok{\textless{}{-}}\NormalTok{ MarkovTransition}\SpecialCharTok{$}\FunctionTok{new}\NormalTok{(s.mono.A, s.mono.A, }\AttributeTok{r=}\ConstantTok{NULL}\NormalTok{)}
\NormalTok{tAB }\OtherTok{\textless{}{-}}\NormalTok{ MarkovTransition}\SpecialCharTok{$}\FunctionTok{new}\NormalTok{(s.mono.A, s.mono.B, }\AttributeTok{r=}\NormalTok{trAB)}
\NormalTok{tAC }\OtherTok{\textless{}{-}}\NormalTok{ MarkovTransition}\SpecialCharTok{$}\FunctionTok{new}\NormalTok{(s.mono.A, s.mono.C, }\AttributeTok{r=}\NormalTok{trAC)}
\NormalTok{tAD }\OtherTok{\textless{}{-}}\NormalTok{ MarkovTransition}\SpecialCharTok{$}\FunctionTok{new}\NormalTok{(s.mono.A, s.mono.D, }\AttributeTok{r=}\NormalTok{trAD)}
\NormalTok{tBB }\OtherTok{\textless{}{-}}\NormalTok{ MarkovTransition}\SpecialCharTok{$}\FunctionTok{new}\NormalTok{(s.mono.B, s.mono.B, }\AttributeTok{r=}\ConstantTok{NULL}\NormalTok{)}
\NormalTok{tBC }\OtherTok{\textless{}{-}}\NormalTok{ MarkovTransition}\SpecialCharTok{$}\FunctionTok{new}\NormalTok{(s.mono.B, s.mono.C, }\AttributeTok{r=}\NormalTok{trBC)}
\NormalTok{tBD }\OtherTok{\textless{}{-}}\NormalTok{ MarkovTransition}\SpecialCharTok{$}\FunctionTok{new}\NormalTok{(s.mono.B, s.mono.D, }\AttributeTok{r=}\NormalTok{trBD)}
\NormalTok{tCC }\OtherTok{\textless{}{-}}\NormalTok{ MarkovTransition}\SpecialCharTok{$}\FunctionTok{new}\NormalTok{(s.mono.C, s.mono.C, }\AttributeTok{r=}\ConstantTok{NULL}\NormalTok{)}
\NormalTok{tCD }\OtherTok{\textless{}{-}}\NormalTok{ MarkovTransition}\SpecialCharTok{$}\FunctionTok{new}\NormalTok{(s.mono.C, s.mono.D, }\AttributeTok{r=}\NormalTok{trCD)}
\NormalTok{tDD }\OtherTok{\textless{}{-}}\NormalTok{ MarkovTransition}\SpecialCharTok{$}\FunctionTok{new}\NormalTok{(s.mono.D, s.mono.D, }\AttributeTok{r=}\ConstantTok{NULL}\NormalTok{)}
\CommentTok{\# construct the model}
\NormalTok{m.mono }\OtherTok{\textless{}{-}}\NormalTok{ CohortMarkovModel}\SpecialCharTok{$}\FunctionTok{new}\NormalTok{(}
  \AttributeTok{V =} \FunctionTok{list}\NormalTok{(s.mono.A, s.mono.B, s.mono.C, s.mono.D),}
  \AttributeTok{E =} \FunctionTok{list}\NormalTok{(tAA, tAB, tAC, tAD, tBB, tBC, tBD, tCC, tCD, tDD),}
  \AttributeTok{discount.cost =}\NormalTok{ cDR}\SpecialCharTok{/}\DecValTok{100}\NormalTok{,}
  \AttributeTok{discount.utility =}\NormalTok{ oDR}\SpecialCharTok{/}\DecValTok{100}
\NormalTok{)}
\end{Highlighting}
\end{Shaded}

\hypertarget{checking-the-model}{%
\section{Checking the model}\label{checking-the-model}}

\hypertarget{diagram}{%
\subsection{Diagram}\label{diagram}}

A representation of the model in DOT format
(\href{https://graphviz.org}{Graphviz}) can be created using the
\texttt{as\_DOT} function of \texttt{CohortMarkovModel}. The function
returns a character vector which can be saved in a file (\texttt{.gv}
extension) for visualization with the \texttt{dot} tool of Graphviz, or
plotted directly in R via the \texttt{DiagrammeR} package. The Markov
model for monotherapy is as follows:

\begin{figure}
\centering
\includegraphics{"mono.png"}
\caption{mono graph}
\end{figure}

\hypertarget{summary-of-model-states}{%
\subsection{Summary of model states}\label{summary-of-model-states}}

\begin{Shaded}
\begin{Highlighting}[]
\CommentTok{\#model.states \textless{}{-} m.mono$stateSummary()}
\end{Highlighting}
\end{Shaded}

\hypertarget{summary-of-annual-transition-probabilities}{%
\subsection{Summary of annual transition
probabilities}\label{summary-of-annual-transition-probabilities}}

\begin{Shaded}
\begin{Highlighting}[]
\CommentTok{\#transition.matrix \textless{}{-} m.mono$transitionSummary()}
\end{Highlighting}
\end{Shaded}

\hypertarget{running-the-model}{%
\section{Running the model}\label{running-the-model}}

\hypertarget{single-cycle}{%
\subsection{Single cycle}\label{single-cycle}}

Model function \texttt{cycle} applies one cycle of a Markov model to a
defined starting population in each state. It returns a table with one
row per state, and each row containing several columns, including the
population at the end of the state, and the cost of occupancy of states,
normalized by the number of patients in the cohort, with discounting
applied. For example, the first cycle of the model is as follows:

\begin{Shaded}
\begin{Highlighting}[]
\CommentTok{\# create starting populations}
\CommentTok{\#populations \textless{}{-} c(\textquotesingle{}A\textquotesingle{}=1000, \textquotesingle{}B\textquotesingle{}=0, \textquotesingle{}C\textquotesingle{}=0, \textquotesingle{}D\textquotesingle{}=0)}
\CommentTok{\#m.mono$setPopulations(populations)}
\CommentTok{\# run the model}
\CommentTok{\#DF \textless{}{-} m.mono$cycle()}
\end{Highlighting}
\end{Shaded}

which returns the following result:

\hypertarget{multiple-cycles}{%
\subsection{Multiple cycles}\label{multiple-cycles}}

Multiple cycles are run by feeding the state populations at the end of
one cycle into the next. Function \texttt{cycles} returns a data frame
with one row per cycle, and each row containing the state populations
and the aggregated cost of occupancy for all states, with discounting
applied. If costs per state, per cycle, are needed, use the lower level
function \texttt{cycle} to extract state values. Below, this is done for
the first 20 cycles of the model. In addition, the proportion of
patients alive at each cycle is added to the table.

\begin{Shaded}
\begin{Highlighting}[]
\CommentTok{\# create starting populations}
\CommentTok{\#N \textless{}{-} 1000}
\CommentTok{\#populations \textless{}{-} c(\textquotesingle{}A\textquotesingle{}=N, \textquotesingle{}B\textquotesingle{}=0, \textquotesingle{}C\textquotesingle{}=0, \textquotesingle{}D\textquotesingle{}=0)}
\CommentTok{\#m.mono$setPopulations(populations)}
\CommentTok{\# run 20 cycles}
\CommentTok{\#DF.mono \textless{}{-} m.mono$cycles(nCycles=20+1)}
\CommentTok{\# calculate the proportion alive at each cycle}
\CommentTok{\#DF.mono$Alive \textless{}{-} (DF.mono$A + DF.mono$B + DF.mono$C)/N}
\end{Highlighting}
\end{Shaded}

This yields the following summary table for monotherapy:

\hypertarget{model-results}{%
\section{Model results}\label{model-results}}

\hypertarget{expected-survival}{%
\subsection{Expected survival}\label{expected-survival}}

The estimated life years is given by summing the proportions of patients
left alive at each cycle.\textsuperscript{1, Exercise 2.5} This is
proved as follows. If patients are assumed to die at the start of the
cycle, then the expected life years is equal to the probability of death
in one cycle multiplied by the survival time. If \(p_i\) is the
proportion of patients alive at the start of cycle \(i\), then the
expected life years is given by \[
\begin{aligned}
E[LY] &= (p_0 - p_1) \times 0 + (p_2 - p_1) \times 1 + \quad ... \quad + (p_{n-1}-p_n)\times (n-1)\\
&= \sum_1^N (p_{i-1} -p_i)\times (i-1)\\
&= \sum_1^N (ip_{i-1}-ip_i -p_{i-1} + p_i)\\
&= \sum_1^N (i-1)p_{i-1} - \sum_1^N i p_i + \sum_1^N p_i\\
&= -Np_n + \sum_1^N p_i\\
\end{aligned}
\] If \(p_N = 0\) (\emph{i.e.} all patients have died by cycle \(N\)),
then \(E[LY] = \sum_1^N p_i\).

\hypertarget{combination-therapy}{%
\subsection{Combination therapy}\label{combination-therapy}}

For combination therapy, the model is constructed as follows:

\begin{Shaded}
\begin{Highlighting}[]
\CommentTok{\# create Markov states for combination therapy (zidovudine and lamivudine)}
\CommentTok{\#state.comb.A \textless{}{-} MarkovState$new("A", dmca+ccca+cAZT+cLam)}
\CommentTok{\#state.comb.B \textless{}{-} MarkovState$new("B", dmcb+cccb+cAZT+cLam)}
\CommentTok{\#state.comb.C \textless{}{-} MarkovState$new("C", dmcc+cccc+cAZT+cLam)}
\CommentTok{\#state.comb.D \textless{}{-} MarkovState$new("D", 0)}
\DocumentationTok{\#\# transition matrix for combination therapy}
\CommentTok{\#I.comb \textless{}{-} matrix(}
\CommentTok{\#  data = c(0.858, 0.103, 0.034, 0.005,}
\CommentTok{\#           0.000, 0.787, 0.207, 0.006,}
\CommentTok{\#           0.000, 0.000, 0.873, 0.127,}
\CommentTok{\#           0.000, 0.000, 0.000, 1.000),}
\CommentTok{\#  nrow = 4,}
\CommentTok{\#  ncol = 4, }
\CommentTok{\#  byrow = T,}
\CommentTok{\#  dimnames = list(c(\textquotesingle{}A\textquotesingle{}, \textquotesingle{}B\textquotesingle{}, \textquotesingle{}C\textquotesingle{}, \textquotesingle{}D\textquotesingle{}), c(\textquotesingle{}A\textquotesingle{}, \textquotesingle{}B\textquotesingle{}, \textquotesingle{}C\textquotesingle{}, \textquotesingle{}D\textquotesingle{}))}
\CommentTok{\#)}
\DocumentationTok{\#\# construct the model}
\CommentTok{\#m.comb \textless{}{-} MarkovModel$new(}
\CommentTok{\#  states = list(state.comb.A, state.comb.B, state.comb.C, state.comb.D),}
\CommentTok{\#  Ip = I.comb,}
\CommentTok{\#  discount = 6.0}
\CommentTok{\#)}
\end{Highlighting}
\end{Shaded}

In this model, lamivudine is given for the first 2 years, with the
treatment effect assumed to persist for the same period. The state
populations and cycle numbers are retained by the model between calls to
\texttt{cycle} or \texttt{cycles} making it easy to change probabilities
or costs during a simulation. Helper functions \texttt{setAnnualCost},
\texttt{setEntryCost} (for a \texttt{MarkovState} object) and
\texttt{setTransitions} (for a \texttt{MarkovModel} object) are provided
for that purpose.

\begin{Shaded}
\begin{Highlighting}[]
\CommentTok{\# run combination therapy model for 2 years}
\CommentTok{\#N \textless{}{-} 1000}
\CommentTok{\#populations \textless{}{-} c(\textquotesingle{}A\textquotesingle{}=N, \textquotesingle{}B\textquotesingle{}=0, \textquotesingle{}C\textquotesingle{}=0, \textquotesingle{}D\textquotesingle{}=0)}
\CommentTok{\#m.comb$setPopulations(populations)}
\CommentTok{\#DF.comb \textless{}{-} m.comb$cycles(nCycles=2+1)}
\DocumentationTok{\#\# revise costs and transitions, and run model for next 18 years}
\CommentTok{\#state.comb.A$setAnnualCost(1701+1055+2278)}
\CommentTok{\#state.comb.B$setAnnualCost(1774+1278+2278)}
\CommentTok{\#state.comb.C$setAnnualCost(6948+2059+2278)}
\CommentTok{\#m.comb$setTransitions(I.mono)}
\CommentTok{\#DF.comb \textless{}{-} rbind(DF.comb, m.comb$cycles(nCycles=18))}
\DocumentationTok{\#\# calculate the proportion alive at end of each cycle}
\CommentTok{\#DF.comb$Alive \textless{}{-} (DF.comb$A + DF.comb$B + DF.comb$C)/N}
\end{Highlighting}
\end{Shaded}

The cycle history for combination therapy is as follows:

\hypertarget{comparison-of-treatments}{%
\subsection{Comparison of treatments}\label{comparison-of-treatments}}

The ICER is calculated by running both models and calculating the
incremental cost per life year gained.

\hypertarget{references}{%
\section*{References}\label{references}}
\addcontentsline{toc}{section}{References}

\hypertarget{refs}{}
\begin{CSLReferences}{0}{0}
\leavevmode\hypertarget{ref-briggs2006}{}%
\CSLLeftMargin{1 }
\CSLRightInline{Briggs A, Claxton K, Sculpher M. \emph{Decision
modelling for health economic evaluation}. {Oxford, UK}: {Oxford
University Press}; 2006.}

\leavevmode\hypertarget{ref-chancellor1997}{}%
\CSLLeftMargin{2 }
\CSLRightInline{Chancellor JV, Hill AM, Sabin CA, Simpson KN, Youle M.
Modelling the cost effectiveness of {Lamivudine}/{Zidovudine}
combination therapy in {HIV} infection. \emph{Pharmacoeconomics}
1997;\textbf{12}:54--66.}

\end{CSLReferences}

\end{document}
