% Options for packages loaded elsewhere
\PassOptionsToPackage{unicode}{hyperref}
\PassOptionsToPackage{hyphens}{url}
%
\documentclass[
]{article}
\usepackage{lmodern}
\usepackage{amsmath}
\usepackage{ifxetex,ifluatex}
\ifnum 0\ifxetex 1\fi\ifluatex 1\fi=0 % if pdftex
  \usepackage[T1]{fontenc}
  \usepackage[utf8]{inputenc}
  \usepackage{textcomp} % provide euro and other symbols
  \usepackage{amssymb}
\else % if luatex or xetex
  \usepackage{unicode-math}
  \defaultfontfeatures{Scale=MatchLowercase}
  \defaultfontfeatures[\rmfamily]{Ligatures=TeX,Scale=1}
\fi
% Use upquote if available, for straight quotes in verbatim environments
\IfFileExists{upquote.sty}{\usepackage{upquote}}{}
\IfFileExists{microtype.sty}{% use microtype if available
  \usepackage[]{microtype}
  \UseMicrotypeSet[protrusion]{basicmath} % disable protrusion for tt fonts
}{}
\makeatletter
\@ifundefined{KOMAClassName}{% if non-KOMA class
  \IfFileExists{parskip.sty}{%
    \usepackage{parskip}
  }{% else
    \setlength{\parindent}{0pt}
    \setlength{\parskip}{6pt plus 2pt minus 1pt}}
}{% if KOMA class
  \KOMAoptions{parskip=half}}
\makeatother
\usepackage{xcolor}
\IfFileExists{xurl.sty}{\usepackage{xurl}}{} % add URL line breaks if available
\IfFileExists{bookmark.sty}{\usepackage{bookmark}}{\usepackage{hyperref}}
\hypersetup{
  pdftitle={A directed graph solution to a New Scientist puzzle},
  pdfauthor={Andrew J. Sims},
  hidelinks,
  pdfcreator={LaTeX via pandoc}}
\urlstyle{same} % disable monospaced font for URLs
\usepackage[margin=1in]{geometry}
\usepackage{color}
\usepackage{fancyvrb}
\newcommand{\VerbBar}{|}
\newcommand{\VERB}{\Verb[commandchars=\\\{\}]}
\DefineVerbatimEnvironment{Highlighting}{Verbatim}{commandchars=\\\{\}}
% Add ',fontsize=\small' for more characters per line
\usepackage{framed}
\definecolor{shadecolor}{RGB}{248,248,248}
\newenvironment{Shaded}{\begin{snugshade}}{\end{snugshade}}
\newcommand{\AlertTok}[1]{\textcolor[rgb]{0.94,0.16,0.16}{#1}}
\newcommand{\AnnotationTok}[1]{\textcolor[rgb]{0.56,0.35,0.01}{\textbf{\textit{#1}}}}
\newcommand{\AttributeTok}[1]{\textcolor[rgb]{0.77,0.63,0.00}{#1}}
\newcommand{\BaseNTok}[1]{\textcolor[rgb]{0.00,0.00,0.81}{#1}}
\newcommand{\BuiltInTok}[1]{#1}
\newcommand{\CharTok}[1]{\textcolor[rgb]{0.31,0.60,0.02}{#1}}
\newcommand{\CommentTok}[1]{\textcolor[rgb]{0.56,0.35,0.01}{\textit{#1}}}
\newcommand{\CommentVarTok}[1]{\textcolor[rgb]{0.56,0.35,0.01}{\textbf{\textit{#1}}}}
\newcommand{\ConstantTok}[1]{\textcolor[rgb]{0.00,0.00,0.00}{#1}}
\newcommand{\ControlFlowTok}[1]{\textcolor[rgb]{0.13,0.29,0.53}{\textbf{#1}}}
\newcommand{\DataTypeTok}[1]{\textcolor[rgb]{0.13,0.29,0.53}{#1}}
\newcommand{\DecValTok}[1]{\textcolor[rgb]{0.00,0.00,0.81}{#1}}
\newcommand{\DocumentationTok}[1]{\textcolor[rgb]{0.56,0.35,0.01}{\textbf{\textit{#1}}}}
\newcommand{\ErrorTok}[1]{\textcolor[rgb]{0.64,0.00,0.00}{\textbf{#1}}}
\newcommand{\ExtensionTok}[1]{#1}
\newcommand{\FloatTok}[1]{\textcolor[rgb]{0.00,0.00,0.81}{#1}}
\newcommand{\FunctionTok}[1]{\textcolor[rgb]{0.00,0.00,0.00}{#1}}
\newcommand{\ImportTok}[1]{#1}
\newcommand{\InformationTok}[1]{\textcolor[rgb]{0.56,0.35,0.01}{\textbf{\textit{#1}}}}
\newcommand{\KeywordTok}[1]{\textcolor[rgb]{0.13,0.29,0.53}{\textbf{#1}}}
\newcommand{\NormalTok}[1]{#1}
\newcommand{\OperatorTok}[1]{\textcolor[rgb]{0.81,0.36,0.00}{\textbf{#1}}}
\newcommand{\OtherTok}[1]{\textcolor[rgb]{0.56,0.35,0.01}{#1}}
\newcommand{\PreprocessorTok}[1]{\textcolor[rgb]{0.56,0.35,0.01}{\textit{#1}}}
\newcommand{\RegionMarkerTok}[1]{#1}
\newcommand{\SpecialCharTok}[1]{\textcolor[rgb]{0.00,0.00,0.00}{#1}}
\newcommand{\SpecialStringTok}[1]{\textcolor[rgb]{0.31,0.60,0.02}{#1}}
\newcommand{\StringTok}[1]{\textcolor[rgb]{0.31,0.60,0.02}{#1}}
\newcommand{\VariableTok}[1]{\textcolor[rgb]{0.00,0.00,0.00}{#1}}
\newcommand{\VerbatimStringTok}[1]{\textcolor[rgb]{0.31,0.60,0.02}{#1}}
\newcommand{\WarningTok}[1]{\textcolor[rgb]{0.56,0.35,0.01}{\textbf{\textit{#1}}}}
\usepackage{longtable,booktabs}
\usepackage{calc} % for calculating minipage widths
% Correct order of tables after \paragraph or \subparagraph
\usepackage{etoolbox}
\makeatletter
\patchcmd\longtable{\par}{\if@noskipsec\mbox{}\fi\par}{}{}
\makeatother
% Allow footnotes in longtable head/foot
\IfFileExists{footnotehyper.sty}{\usepackage{footnotehyper}}{\usepackage{footnote}}
\makesavenoteenv{longtable}
\usepackage{graphicx}
\makeatletter
\def\maxwidth{\ifdim\Gin@nat@width>\linewidth\linewidth\else\Gin@nat@width\fi}
\def\maxheight{\ifdim\Gin@nat@height>\textheight\textheight\else\Gin@nat@height\fi}
\makeatother
% Scale images if necessary, so that they will not overflow the page
% margins by default, and it is still possible to overwrite the defaults
% using explicit options in \includegraphics[width, height, ...]{}
\setkeys{Gin}{width=\maxwidth,height=\maxheight,keepaspectratio}
% Set default figure placement to htbp
\makeatletter
\def\fps@figure{htbp}
\makeatother
\setlength{\emergencystretch}{3em} % prevent overfull lines
\providecommand{\tightlist}{%
  \setlength{\itemsep}{0pt}\setlength{\parskip}{0pt}}
\setcounter{secnumdepth}{-\maxdimen} % remove section numbering
\ifluatex
  \usepackage{selnolig}  % disable illegal ligatures
\fi
\newlength{\cslhangindent}
\setlength{\cslhangindent}{1.5em}
\newlength{\csllabelwidth}
\setlength{\csllabelwidth}{3em}
\newenvironment{CSLReferences}[2] % #1 hanging-ident, #2 entry spacing
 {% don't indent paragraphs
  \setlength{\parindent}{0pt}
  % turn on hanging indent if param 1 is 1
  \ifodd #1 \everypar{\setlength{\hangindent}{\cslhangindent}}\ignorespaces\fi
  % set entry spacing
  \ifnum #2 > 0
  \setlength{\parskip}{#2\baselineskip}
  \fi
 }%
 {}
\usepackage{calc}
\newcommand{\CSLBlock}[1]{#1\hfill\break}
\newcommand{\CSLLeftMargin}[1]{\parbox[t]{\csllabelwidth}{#1}}
\newcommand{\CSLRightInline}[1]{\parbox[t]{\linewidth - \csllabelwidth}{#1}\break}
\newcommand{\CSLIndent}[1]{\hspace{\cslhangindent}#1}

\title{A directed graph solution to a \emph{New Scientist} puzzle}
\usepackage{etoolbox}
\makeatletter
\providecommand{\subtitle}[1]{% add subtitle to \maketitle
  \apptocmd{\@title}{\par {\large #1 \par}}{}{}
}
\makeatother
\subtitle{Burger run}
\author{Andrew J. Sims}
\date{18th June 2020}

\begin{document}
\maketitle

\hypertarget{introduction}{%
\section{Introduction}\label{introduction}}

This puzzle was published in \emph{New Scientist} in June 2020 {[}1{]}.
It is a practical example of a problem in graph theory. This vignette
explains how the puzzle can be solved with \texttt{redecison}.

\hypertarget{the-puzzle}{%
\section{The puzzle}\label{the-puzzle}}

Three friends agree to drive from A to B via the shortest road possible
(driving down or right at all times). They are hungry, so also want to
drive through a Big Burger restaurant, marked in red. They are arguing
about how many shortest routes will pass through exactly one Big Burger.
Xenia: ``I reckon there are 10.'' Yolanda: ``I'd say more like 20.''
Zara: ``No you're both wrong, I bet there are more than 50.'' Who is
right, or closest to right?

\begin{center}\includegraphics[width=8cm]{GT01-NewScientistPuzzle_files/figure-latex/diagram-1} \end{center}

\hypertarget{constructing-the-graph}{%
\section{Constructing the graph}\label{constructing-the-graph}}

The grid has 25 nodes and 40 edges (20 horizontal and 20 vertical).
These form a directed graph because it is allowed to drive down or right
only. Seven of the edges are defined as ``Big Burger'' edges. Because it
is not possible to find a path from any node which revisits that node,
the graph is acyclic (a directed acyclic graph, DAG).

Although it possible to construct the graph by creating 25 node objects
explicitly, it is more compact to create a list of vertices in a loop
construct. Indices \(i = [1 .. 5]\) and \(j = [1 .. 5]\) are used to
identify grid intersections in the vertical and horizontal directions
respectively. Each node is labelled as \(N_{i,j}\) and the index of node
\(N_{i,j}\) in the list is \(5(i-1)+j\).

Similarly, the 40 edges (arrows) are constructed more compactly in a
list, with horizontal edges being labelled \(H_{i,j}\) (the horizontal
edge joining node \(N_{i,j}\) to node \(N_{i,j+1}\)) and the vertical
edges similarly as \(V_{i,j}\).

\begin{Shaded}
\begin{Highlighting}[]
\CommentTok{\# create vertices}
\NormalTok{V }\OtherTok{\textless{}{-}} \FunctionTok{list}\NormalTok{()}
\ControlFlowTok{for}\NormalTok{ (i }\ControlFlowTok{in} \DecValTok{1}\SpecialCharTok{:}\DecValTok{5}\NormalTok{) \{}
  \ControlFlowTok{for}\NormalTok{ (j }\ControlFlowTok{in} \DecValTok{1}\SpecialCharTok{:}\DecValTok{5}\NormalTok{) \{}
\NormalTok{    V }\OtherTok{\textless{}{-}} \FunctionTok{c}\NormalTok{(V, Node}\SpecialCharTok{$}\FunctionTok{new}\NormalTok{(}\FunctionTok{paste}\NormalTok{(}\StringTok{"N"}\NormalTok{,i,j,}\AttributeTok{sep=}\StringTok{""}\NormalTok{)))}
\NormalTok{  \}}
\NormalTok{\}}
\CommentTok{\# create edges}
\NormalTok{E }\OtherTok{\textless{}{-}} \FunctionTok{list}\NormalTok{()}
\ControlFlowTok{for}\NormalTok{ (i }\ControlFlowTok{in} \DecValTok{1}\SpecialCharTok{:}\DecValTok{5}\NormalTok{) \{}
  \ControlFlowTok{for}\NormalTok{ (j }\ControlFlowTok{in} \DecValTok{1}\SpecialCharTok{:}\DecValTok{4}\NormalTok{) \{}
\NormalTok{    E }\OtherTok{\textless{}{-}} \FunctionTok{c}\NormalTok{(E, Arrow}\SpecialCharTok{$}\FunctionTok{new}\NormalTok{(V[[}\DecValTok{5}\SpecialCharTok{*}\NormalTok{(i}\DecValTok{{-}1}\NormalTok{)}\SpecialCharTok{+}\NormalTok{j]], V[[}\DecValTok{5}\SpecialCharTok{*}\NormalTok{(i}\DecValTok{{-}1}\NormalTok{)}\SpecialCharTok{+}\NormalTok{j}\SpecialCharTok{+}\DecValTok{1}\NormalTok{]], }\FunctionTok{paste}\NormalTok{(}\StringTok{"H"}\NormalTok{,i,j,}\AttributeTok{sep=}\StringTok{""}\NormalTok{)))}
\NormalTok{  \}}
\NormalTok{\} }
\ControlFlowTok{for}\NormalTok{ (i }\ControlFlowTok{in} \DecValTok{1}\SpecialCharTok{:}\DecValTok{4}\NormalTok{) \{}
  \ControlFlowTok{for}\NormalTok{ (j }\ControlFlowTok{in} \DecValTok{1}\SpecialCharTok{:}\DecValTok{5}\NormalTok{) \{}
\NormalTok{    E }\OtherTok{\textless{}{-}} \FunctionTok{c}\NormalTok{(E, Arrow}\SpecialCharTok{$}\FunctionTok{new}\NormalTok{(V[[}\DecValTok{5}\SpecialCharTok{*}\NormalTok{(i}\DecValTok{{-}1}\NormalTok{)}\SpecialCharTok{+}\NormalTok{j]], V[[}\DecValTok{5}\SpecialCharTok{*}\NormalTok{i}\SpecialCharTok{+}\NormalTok{j]], }\FunctionTok{paste}\NormalTok{(}\StringTok{"V"}\NormalTok{,i,j,}\AttributeTok{sep=}\StringTok{""}\NormalTok{)))}
\NormalTok{  \}}
\NormalTok{\} }
\CommentTok{\# create graph}
\NormalTok{G }\OtherTok{\textless{}{-}}\NormalTok{ Digraph}\SpecialCharTok{$}\FunctionTok{new}\NormalTok{(V,E)}
\end{Highlighting}
\end{Shaded}

\hypertarget{finding-the-paths}{%
\section{Finding the paths}\label{finding-the-paths}}

Method \texttt{paths} finds all possible paths between any two nodes,
where a \emph{path} is defined as a sequence of distinct and adjacent
nodes. Because the restaurants are specific edges, each path is
converted to a \emph{walk}, which is a path defined as sequence of
connected, non-repeating edges.

In this case, the number of restaurants traversed by each path is
counted by comparing the label associated with each edge in each path
with the labels of the edges which contain a restaurant.

\begin{Shaded}
\begin{Highlighting}[]
\CommentTok{\# get all paths from A to B}
\NormalTok{A }\OtherTok{\textless{}{-}}\NormalTok{ V[[}\DecValTok{1}\NormalTok{]]}
\NormalTok{B }\OtherTok{\textless{}{-}}\NormalTok{ V[[}\DecValTok{25}\NormalTok{]]}
\NormalTok{P }\OtherTok{\textless{}{-}}\NormalTok{ G}\SpecialCharTok{$}\FunctionTok{paths}\NormalTok{(A,B)}
\CommentTok{\# convert paths to walks}
\NormalTok{W }\OtherTok{\textless{}{-}} \FunctionTok{lapply}\NormalTok{(P,}\ControlFlowTok{function}\NormalTok{(p)\{G}\SpecialCharTok{$}\FunctionTok{walk}\NormalTok{(p)\})}
\CommentTok{\# count and tabulate how many special edges each walk traverses}
\NormalTok{BB }\OtherTok{\textless{}{-}} \FunctionTok{c}\NormalTok{(}\StringTok{"V11"}\NormalTok{, }\StringTok{"H22"}\NormalTok{, }\StringTok{"V25"}\NormalTok{, }\StringTok{"H33"}\NormalTok{, }\StringTok{"V32"}\NormalTok{, }\StringTok{"H44"}\NormalTok{, }\StringTok{"V43"}\NormalTok{)}
\NormalTok{nw }\OtherTok{\textless{}{-}} \FunctionTok{sapply}\NormalTok{(W, }\ControlFlowTok{function}\NormalTok{(w) \{}
\NormalTok{  lv }\OtherTok{\textless{}{-}} \FunctionTok{sapply}\NormalTok{(w, }\ControlFlowTok{function}\NormalTok{(e) \{e}\SpecialCharTok{$}\FunctionTok{label}\NormalTok{() }\SpecialCharTok{\%in\%}\NormalTok{ BB\}) }
  \FunctionTok{return}\NormalTok{(}\FunctionTok{sum}\NormalTok{(lv))}
\NormalTok{\})}
\CommentTok{\# tabulate }
\NormalTok{ct }\OtherTok{\textless{}{-}} \FunctionTok{as.data.frame}\NormalTok{(}\FunctionTok{table}\NormalTok{(nw))}
\end{Highlighting}
\end{Shaded}

\hypertarget{solution-found-by-rdecision}{%
\section{\texorpdfstring{Solution found by
\texttt{rdecision}}{Solution found by rdecision}}\label{solution-found-by-rdecision}}

The number of paths which pass through exactly one Big Burger is 23. In
total there are 70 paths from A to B, with the number of restaurants
\(n\), traversed by each path as follows:

\begin{longtable}[]{@{}lr@{}}
\toprule
n & frequency\tabularnewline
\midrule
\endhead
0 & 6\tabularnewline
1 & 23\tabularnewline
2 & 27\tabularnewline
3 & 13\tabularnewline
4 & 1\tabularnewline
\bottomrule
\end{longtable}

\hypertarget{provided-solution}{%
\section{Provided solution}\label{provided-solution}}

Yolanda's estimate is closest - there are 23 shortest routes from A to B
that pass through exactly one Big Burger. One way to solve this kind of
puzzle is to systematically work from A and keep track of how many ways
there are of reaching each point. With this problem, you should keep a
separate count of how many ways there are of reaching each point after
(a) zero or (b) one Big Burger visits. For line segments that contain a
Big Burger, (b) becomes equal to (a) then becomes equal to 0 with the
old value for (b) effectively discarded.

\hypertarget{references}{%
\section*{References}\label{references}}
\addcontentsline{toc}{section}{References}

\hypertarget{refs}{}
\begin{CSLReferences}{0}{0}
\leavevmode\hypertarget{ref-bodycombe2020}{}%
\CSLLeftMargin{1 }
\CSLRightInline{Bodycombe D. Burger run. \emph{New Scientist}
2020;\textbf{246}:54.}

\end{CSLReferences}

\end{document}
