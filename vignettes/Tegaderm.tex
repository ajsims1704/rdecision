\PassOptionsToPackage{unicode=true}{hyperref} % options for packages loaded elsewhere
\PassOptionsToPackage{hyphens}{url}
%
\documentclass[]{article}
\usepackage{lmodern}
\usepackage{amssymb,amsmath}
\usepackage{ifxetex,ifluatex}
\usepackage{fixltx2e} % provides \textsubscript
\ifnum 0\ifxetex 1\fi\ifluatex 1\fi=0 % if pdftex
  \usepackage[T1]{fontenc}
  \usepackage[utf8]{inputenc}
  \usepackage{textcomp} % provides euro and other symbols
\else % if luatex or xelatex
  \usepackage{unicode-math}
  \defaultfontfeatures{Ligatures=TeX,Scale=MatchLowercase}
\fi
% use upquote if available, for straight quotes in verbatim environments
\IfFileExists{upquote.sty}{\usepackage{upquote}}{}
% use microtype if available
\IfFileExists{microtype.sty}{%
\usepackage[]{microtype}
\UseMicrotypeSet[protrusion]{basicmath} % disable protrusion for tt fonts
}{}
\IfFileExists{parskip.sty}{%
\usepackage{parskip}
}{% else
\setlength{\parindent}{0pt}
\setlength{\parskip}{6pt plus 2pt minus 1pt}
}
\usepackage{hyperref}
\hypersetup{
            pdftitle={Tegaderm CHG IV Securement Dressing for Central Venous and Arterial Catheter Insertion Sites},
            pdfauthor={Andrew J. Sims},
            pdfborder={0 0 0},
            breaklinks=true}
\urlstyle{same}  % don't use monospace font for urls
\usepackage[margin=1in]{geometry}
\usepackage{color}
\usepackage{fancyvrb}
\newcommand{\VerbBar}{|}
\newcommand{\VERB}{\Verb[commandchars=\\\{\}]}
\DefineVerbatimEnvironment{Highlighting}{Verbatim}{commandchars=\\\{\}}
% Add ',fontsize=\small' for more characters per line
\usepackage{framed}
\definecolor{shadecolor}{RGB}{248,248,248}
\newenvironment{Shaded}{\begin{snugshade}}{\end{snugshade}}
\newcommand{\AlertTok}[1]{\textcolor[rgb]{0.94,0.16,0.16}{#1}}
\newcommand{\AnnotationTok}[1]{\textcolor[rgb]{0.56,0.35,0.01}{\textbf{\textit{#1}}}}
\newcommand{\AttributeTok}[1]{\textcolor[rgb]{0.77,0.63,0.00}{#1}}
\newcommand{\BaseNTok}[1]{\textcolor[rgb]{0.00,0.00,0.81}{#1}}
\newcommand{\BuiltInTok}[1]{#1}
\newcommand{\CharTok}[1]{\textcolor[rgb]{0.31,0.60,0.02}{#1}}
\newcommand{\CommentTok}[1]{\textcolor[rgb]{0.56,0.35,0.01}{\textit{#1}}}
\newcommand{\CommentVarTok}[1]{\textcolor[rgb]{0.56,0.35,0.01}{\textbf{\textit{#1}}}}
\newcommand{\ConstantTok}[1]{\textcolor[rgb]{0.00,0.00,0.00}{#1}}
\newcommand{\ControlFlowTok}[1]{\textcolor[rgb]{0.13,0.29,0.53}{\textbf{#1}}}
\newcommand{\DataTypeTok}[1]{\textcolor[rgb]{0.13,0.29,0.53}{#1}}
\newcommand{\DecValTok}[1]{\textcolor[rgb]{0.00,0.00,0.81}{#1}}
\newcommand{\DocumentationTok}[1]{\textcolor[rgb]{0.56,0.35,0.01}{\textbf{\textit{#1}}}}
\newcommand{\ErrorTok}[1]{\textcolor[rgb]{0.64,0.00,0.00}{\textbf{#1}}}
\newcommand{\ExtensionTok}[1]{#1}
\newcommand{\FloatTok}[1]{\textcolor[rgb]{0.00,0.00,0.81}{#1}}
\newcommand{\FunctionTok}[1]{\textcolor[rgb]{0.00,0.00,0.00}{#1}}
\newcommand{\ImportTok}[1]{#1}
\newcommand{\InformationTok}[1]{\textcolor[rgb]{0.56,0.35,0.01}{\textbf{\textit{#1}}}}
\newcommand{\KeywordTok}[1]{\textcolor[rgb]{0.13,0.29,0.53}{\textbf{#1}}}
\newcommand{\NormalTok}[1]{#1}
\newcommand{\OperatorTok}[1]{\textcolor[rgb]{0.81,0.36,0.00}{\textbf{#1}}}
\newcommand{\OtherTok}[1]{\textcolor[rgb]{0.56,0.35,0.01}{#1}}
\newcommand{\PreprocessorTok}[1]{\textcolor[rgb]{0.56,0.35,0.01}{\textit{#1}}}
\newcommand{\RegionMarkerTok}[1]{#1}
\newcommand{\SpecialCharTok}[1]{\textcolor[rgb]{0.00,0.00,0.00}{#1}}
\newcommand{\SpecialStringTok}[1]{\textcolor[rgb]{0.31,0.60,0.02}{#1}}
\newcommand{\StringTok}[1]{\textcolor[rgb]{0.31,0.60,0.02}{#1}}
\newcommand{\VariableTok}[1]{\textcolor[rgb]{0.00,0.00,0.00}{#1}}
\newcommand{\VerbatimStringTok}[1]{\textcolor[rgb]{0.31,0.60,0.02}{#1}}
\newcommand{\WarningTok}[1]{\textcolor[rgb]{0.56,0.35,0.01}{\textbf{\textit{#1}}}}
\usepackage{longtable,booktabs}
% Fix footnotes in tables (requires footnote package)
\IfFileExists{footnote.sty}{\usepackage{footnote}\makesavenoteenv{longtable}}{}
\usepackage{graphicx,grffile}
\makeatletter
\def\maxwidth{\ifdim\Gin@nat@width>\linewidth\linewidth\else\Gin@nat@width\fi}
\def\maxheight{\ifdim\Gin@nat@height>\textheight\textheight\else\Gin@nat@height\fi}
\makeatother
% Scale images if necessary, so that they will not overflow the page
% margins by default, and it is still possible to overwrite the defaults
% using explicit options in \includegraphics[width, height, ...]{}
\setkeys{Gin}{width=\maxwidth,height=\maxheight,keepaspectratio}
\setlength{\emergencystretch}{3em}  % prevent overfull lines
\providecommand{\tightlist}{%
  \setlength{\itemsep}{0pt}\setlength{\parskip}{0pt}}
\setcounter{secnumdepth}{0}
% Redefines (sub)paragraphs to behave more like sections
\ifx\paragraph\undefined\else
\let\oldparagraph\paragraph
\renewcommand{\paragraph}[1]{\oldparagraph{#1}\mbox{}}
\fi
\ifx\subparagraph\undefined\else
\let\oldsubparagraph\subparagraph
\renewcommand{\subparagraph}[1]{\oldsubparagraph{#1}\mbox{}}
\fi

% set default figure placement to htbp
\makeatletter
\def\fps@figure{htbp}
\makeatother

\usepackage{etoolbox}
\makeatletter
\providecommand{\subtitle}[1]{% add subtitle to \maketitle
  \apptocmd{\@title}{\par {\large #1 \par}}{}{}
}
\makeatother

\title{Tegaderm CHG IV Securement Dressing for Central Venous and Arterial
Catheter Insertion Sites}
\providecommand{\subtitle}[1]{}
\subtitle{A decision tree example with probabilistic sensitivity analysis}
\author{Andrew J. Sims}
\date{2020-05-30}

\begin{document}
\maketitle

\hypertarget{introduction}{%
\section{Introduction}\label{introduction}}

This vignette is an example of modelling a decision tree using the
\texttt{rdecision} package, with probabilistic sensitivity analysis. It
is based on the model reported by Jenks \emph{et al}\textsuperscript{1}
in which a transparent dressing used to secure vascular catheters
(Tegaderm CHG) was compared with a standard dressing.

Two methods of evaluating the decision are presented. The first method
constructs a decision tree and evaluates the costs associated with
traversing each pathway through it. The second method is a direct
calculation and summation of costs, without the need to construct a
tree. Point estimates and probabilistic sensitivity analysis are
conducted for both methods.

\hypertarget{model-variables}{%
\section{Model variables}\label{model-variables}}

Thirteen variables were used in the model. The choice of variables,
their distributions and their parameters are taken from table 3 of Jenks
\emph{et al}\textsuperscript{1}, with the following corrections:

\begin{itemize}
\tightlist
\item
  For variables with lognormal uncertainty, the manufacturer gave values
  for the mean \(m\) and standard deviation \(s\) in log space. However,
  their standard deviations were quoted as negative values. This was an
  error, but had no effect on their results, because they sampled values
  of \(\exp(m + sz)\), where \(z\) is a sample from a standard normal
  distribution and is symmetrical about 0. For the variables with log
  normal uncertainty given below, positive standard deviation
  parameters, with the same absolute value, have been used as
  hyperparameters of the log normal distributions. Also note that the
  \emph{median} value on the natural scale of a random variable
  distributed as \(logN(m,s)\), where \(\mu\) and \(\sigma\) are the
  mean and standard deviation on the log scale, is \(e^\mu\); the mean
  on the natural scale is slightly larger. For example, the hazard ratio
  for CRBSI with Tegaderm versus standard dressing was modelled as
  \(logN(-0.911,0.393)\), which has median 0.402 (the point estimate of
  the ratio from the literature) and mean 0.434.
\item
  The relative risk for dermatitis was modelled as
  \(logN(1.482,0.490)\).
\item
  The point estimate cost of CRBSI was £9900, not £9990, although the
  parameters (198,50) are quoted correctly.
\end{itemize}

The 13 model variables were constructed as follows:

\begin{Shaded}
\begin{Highlighting}[]
\CommentTok{# clinical variables}
\NormalTok{r.CRBSI <-}\StringTok{ }\NormalTok{NormModVar}\OperatorTok{$}\KeywordTok{new}\NormalTok{(}
  \StringTok{'Baseline CRBSI rate'}\NormalTok{, }\StringTok{'/1000 catheter days'}\NormalTok{, }\DataTypeTok{mu=}\FloatTok{1.48}\NormalTok{, }\DataTypeTok{sigma=}\FloatTok{0.074}
\NormalTok{)}
\NormalTok{hr.CRBSI <-}\StringTok{ }\NormalTok{LogNormModVar}\OperatorTok{$}\KeywordTok{new}\NormalTok{(}
  \StringTok{'Tegaderm CRBSI HR'}\NormalTok{, }\StringTok{'ratio'}\NormalTok{, }\DataTypeTok{p1=}\OperatorTok{-}\FloatTok{0.911}\NormalTok{, }\DataTypeTok{p2=}\FloatTok{0.393}
\NormalTok{)}
\NormalTok{r.LSI <-}\StringTok{ }\NormalTok{NormModVar}\OperatorTok{$}\KeywordTok{new}\NormalTok{(}
  \StringTok{'Baseline LSI rate'}\NormalTok{, }\StringTok{'/patient'}\NormalTok{, }\DataTypeTok{mu=}\FloatTok{0.1}\NormalTok{, }\DataTypeTok{sigma=}\FloatTok{0.01}
\NormalTok{)}
\NormalTok{hr.LSI <-}\StringTok{ }\NormalTok{LogNormModVar}\OperatorTok{$}\KeywordTok{new}\NormalTok{(}
  \StringTok{'Tegaderm LSI HR'}\NormalTok{, }\StringTok{'ratio'}\NormalTok{, }\DataTypeTok{p1=}\OperatorTok{-}\FloatTok{0.911}\NormalTok{, }\DataTypeTok{p2=}\FloatTok{0.393}
\NormalTok{)}
\NormalTok{r.Dermatitis <-}\StringTok{ }\NormalTok{NormModVar}\OperatorTok{$}\KeywordTok{new}\NormalTok{(}
  \StringTok{'Baseline dermatitis risk'}\NormalTok{, }\StringTok{'/catheter'}\NormalTok{, }\DataTypeTok{mu=}\FloatTok{0.0026}\NormalTok{, }\DataTypeTok{sigma=}\FloatTok{0.00026}
\NormalTok{)}
\NormalTok{rr.Dermatitis <-}\StringTok{ }\NormalTok{LogNormModVar}\OperatorTok{$}\KeywordTok{new}\NormalTok{(}
  \StringTok{'Tegaderm Dermatitis RR'}\NormalTok{, }\StringTok{'ratio'}\NormalTok{,  }\DataTypeTok{p1=}\FloatTok{1.482}\NormalTok{, }\DataTypeTok{p2=}\FloatTok{0.490}
\NormalTok{)}

\CommentTok{# cost variables}
\NormalTok{c.CRBSI <-}\StringTok{ }\NormalTok{GammaModVar}\OperatorTok{$}\KeywordTok{new}\NormalTok{(}
  \StringTok{'CRBSI cost'}\NormalTok{, }\StringTok{'GBP'}\NormalTok{, }\DataTypeTok{alpha=}\FloatTok{198.0}\NormalTok{, }\DataTypeTok{beta=}\DecValTok{50}
\NormalTok{)}
\NormalTok{c.Dermatitis <-}\StringTok{ }\NormalTok{GammaModVar}\OperatorTok{$}\KeywordTok{new}\NormalTok{(}
  \StringTok{'Dermatitis cost'}\NormalTok{, }\StringTok{'GBP'}\NormalTok{, }\DataTypeTok{alpha=}\DecValTok{30}\NormalTok{, }\DataTypeTok{beta=}\DecValTok{5}
\NormalTok{)}
\NormalTok{c.LSI <-}\StringTok{ }\NormalTok{GammaModVar}\OperatorTok{$}\KeywordTok{new}\NormalTok{(}
  \StringTok{'LSI cost'}\NormalTok{, }\StringTok{'GBP'}\NormalTok{, }\DataTypeTok{alpha=}\DecValTok{50}\NormalTok{, }\DataTypeTok{beta=}\DecValTok{5}
\NormalTok{)}
\NormalTok{c.Tegaderm <-}\StringTok{ }\NormalTok{ConstModVar}\OperatorTok{$}\KeywordTok{new}\NormalTok{(}
  \StringTok{'Tegaderm CHG cost'}\NormalTok{, }\StringTok{'GBP'}\NormalTok{, }\DataTypeTok{const=}\FloatTok{6.21}
\NormalTok{)}
\NormalTok{c.Standard <-}\StringTok{ }\NormalTok{ConstModVar}\OperatorTok{$}\KeywordTok{new}\NormalTok{(}
  \StringTok{'Standard dressing cost'}\NormalTok{, }\StringTok{'GBP'}\NormalTok{, }\DataTypeTok{const=}\FloatTok{1.34}
\NormalTok{)}
\NormalTok{n.cathdays <-}\StringTok{ }\NormalTok{NormModVar}\OperatorTok{$}\KeywordTok{new}\NormalTok{(}
  \StringTok{'No. days with catheter'}\NormalTok{, }\StringTok{'days'}\NormalTok{, }\DataTypeTok{mu=}\DecValTok{10}\NormalTok{, }\DataTypeTok{sigma=}\DecValTok{2}
\NormalTok{)}
\NormalTok{n.dressings <-}\StringTok{ }\NormalTok{NormModVar}\OperatorTok{$}\KeywordTok{new}\NormalTok{(}
  \StringTok{'No. dressings'}\NormalTok{, }\StringTok{'dressings'}\NormalTok{, }\DataTypeTok{mu=}\DecValTok{3}\NormalTok{, }\DataTypeTok{sigma=}\FloatTok{0.3}
\NormalTok{)}
\end{Highlighting}
\end{Shaded}

\hypertarget{the-decision-tree-approach}{%
\section{The decision tree approach}\label{the-decision-tree-approach}}

The decision problem may be solved by constructing a decision tree
comprising decision nodes, chance nodes and leaf nodes. The general
approach is to create expressions involving model variables, construct a
tree, and then evaluate the decision for the base case and its
uncertainty.

\hypertarget{model-variable-expressions}{%
\subsection{Model variable
expressions}\label{model-variable-expressions}}

Variables in the model may be included in the decision tree via
mathematical expressions, which involve model variables and are
themselves model variables. Forms of expression involving R's numerical
functions and multiple model variables are supported, provided they
conform to R syntax. The following code creates the model variable
expressions to be used as values in the decision tree nodes.

\begin{Shaded}
\begin{Highlighting}[]
\CommentTok{# probabilities}
\NormalTok{p.Dermatitis.S <-}\StringTok{ }\NormalTok{ExprModVar}\OperatorTok{$}\KeywordTok{new}\NormalTok{(}
  \StringTok{'P(dermatitis|standard dressing)'}\NormalTok{, }\StringTok{'P'}\NormalTok{, }\KeywordTok{quote}\NormalTok{(n.dressings}\OperatorTok{*}\NormalTok{r.Dermatitis)}
\NormalTok{)}
\NormalTok{p.Dermatitis.T <-}\StringTok{ }\NormalTok{ExprModVar}\OperatorTok{$}\KeywordTok{new}\NormalTok{(}
  \StringTok{'P(dermatitis|Tegaderm)'}\NormalTok{, }\StringTok{'P'}\NormalTok{, }\KeywordTok{quote}\NormalTok{(n.dressings}\OperatorTok{*}\NormalTok{r.Dermatitis}\OperatorTok{*}\NormalTok{rr.Dermatitis)}
\NormalTok{)}
\NormalTok{r.LSI.T <-}\StringTok{ }\NormalTok{ExprModVar}\OperatorTok{$}\KeywordTok{new}\NormalTok{(}
  \StringTok{'P(LSI|Tegaderm)'}\NormalTok{, }\StringTok{'P'}\NormalTok{, }\KeywordTok{quote}\NormalTok{(r.LSI}\OperatorTok{*}\NormalTok{hr.LSI)}
\NormalTok{)}
\NormalTok{p.CRBSI.S <-}\StringTok{ }\NormalTok{ExprModVar}\OperatorTok{$}\KeywordTok{new}\NormalTok{(}
  \StringTok{'P(CRBSI|standard dressing)'}\NormalTok{, }\StringTok{'P'}\NormalTok{,  }\KeywordTok{quote}\NormalTok{(r.CRBSI}\OperatorTok{*}\NormalTok{n.cathdays}\OperatorTok{/}\DecValTok{1000}\NormalTok{)}
\NormalTok{)}
\NormalTok{p.CRBSI.T <-}\StringTok{ }\NormalTok{ExprModVar}\OperatorTok{$}\KeywordTok{new}\NormalTok{(}
  \StringTok{'P(CRBSI|Tegaderm)'}\NormalTok{, }\StringTok{'P'}\NormalTok{, }\KeywordTok{quote}\NormalTok{(r.CRBSI}\OperatorTok{*}\NormalTok{n.cathdays}\OperatorTok{*}\NormalTok{hr.CRBSI}\OperatorTok{/}\DecValTok{1000}\NormalTok{)}
\NormalTok{)}

\CommentTok{# costs to each branch}
\NormalTok{c.S <-}\StringTok{ }\NormalTok{ExprModVar}\OperatorTok{$}\KeywordTok{new}\NormalTok{(}
  \StringTok{'Cost of standard dressing'}\NormalTok{, }\StringTok{'GBP'}\NormalTok{, }\KeywordTok{quote}\NormalTok{(n.dressings}\OperatorTok{*}\NormalTok{c.Standard)}
\NormalTok{)}
\NormalTok{c.T <-}\StringTok{ }\NormalTok{ExprModVar}\OperatorTok{$}\KeywordTok{new}\NormalTok{(}
  \StringTok{'Cost of Tegaderm'}\NormalTok{, }\StringTok{'GBP'}\NormalTok{, }\KeywordTok{quote}\NormalTok{(n.dressings}\OperatorTok{*}\NormalTok{c.Tegaderm)}
\NormalTok{)}
\NormalTok{c.Dermatitis.S <-}\StringTok{ }\NormalTok{ExprModVar}\OperatorTok{$}\KeywordTok{new}\NormalTok{(}\StringTok{"c(Dermatitis,Std)"}\NormalTok{, }\StringTok{"GBP"}\NormalTok{, }\KeywordTok{quote}\NormalTok{(c.Dermatitis}\OperatorTok{+}\NormalTok{c.S))}
\NormalTok{c.Dermatitis.T <-}\StringTok{ }\NormalTok{ExprModVar}\OperatorTok{$}\KeywordTok{new}\NormalTok{(}\StringTok{"c(Dermatitis,Teg)"}\NormalTok{, }\StringTok{"GBP"}\NormalTok{, }\KeywordTok{quote}\NormalTok{(c.Dermatitis}\OperatorTok{+}\NormalTok{c.T))}
\NormalTok{c.LSI.S <-}\StringTok{ }\NormalTok{ExprModVar}\OperatorTok{$}\KeywordTok{new}\NormalTok{(}\StringTok{"c(LSI,Std)"}\NormalTok{, }\StringTok{"GBP"}\NormalTok{, }\KeywordTok{quote}\NormalTok{(c.LSI}\OperatorTok{+}\NormalTok{c.S))}
\NormalTok{c.LSI.T <-}\StringTok{ }\NormalTok{ExprModVar}\OperatorTok{$}\KeywordTok{new}\NormalTok{(}\StringTok{"c(LSI,Teg)"}\NormalTok{, }\StringTok{"GBP"}\NormalTok{, }\KeywordTok{quote}\NormalTok{(c.LSI}\OperatorTok{+}\NormalTok{c.T))}
\NormalTok{c.CRBSI.S <-}\StringTok{ }\NormalTok{ExprModVar}\OperatorTok{$}\KeywordTok{new}\NormalTok{(}\StringTok{"c(CRBSI),Std)"}\NormalTok{, }\StringTok{"GBP"}\NormalTok{, }\KeywordTok{quote}\NormalTok{(c.CRBSI}\OperatorTok{+}\NormalTok{c.S))}
\NormalTok{c.CRBSI.T <-}\StringTok{ }\NormalTok{ExprModVar}\OperatorTok{$}\KeywordTok{new}\NormalTok{(}\StringTok{"c(CRBSI),Teg)"}\NormalTok{, }\StringTok{"GBP"}\NormalTok{, }\KeywordTok{quote}\NormalTok{(c.CRBSI}\OperatorTok{+}\NormalTok{c.T))}
\end{Highlighting}
\end{Shaded}

\hypertarget{the-decision-tree}{%
\subsection{The decision tree}\label{the-decision-tree}}

The following code constructs the decision tree, node by node, based on
figure 2 of Jenks \emph{et al}\textsuperscript{1}. In the formulation
used by \texttt{rdecision}, each node is a potentially recursive
structure which is allowed to have zero or more child nodes; any child
nodes must have already been declared before their parent node is
declared. This implies that a tree should be constructed from right to
left, starting with leaf nodes which have no children (leaf nodes are
synonymous with pathways in Briggs' terminology\textsuperscript{2}). The
final node to be constructed is the node representing the decision
problem. The time horizon is not stated explicitly in the model, and is
assumed to be 7 days here. It was implied that the time horizon was ICU
stay plus some follow-up, and the costs reflect those incurred in that
period, so the assumption of 7 days does not affect the
\texttt{rdecision} implementation of the model.

\begin{Shaded}
\begin{Highlighting}[]
\CommentTok{# Time horizon}
\NormalTok{th <-}\StringTok{ }\KeywordTok{as.difftime}\NormalTok{(}\DecValTok{7}\NormalTok{, }\DataTypeTok{units=}\StringTok{"days"}\NormalTok{)}

\CommentTok{# standard dressing branch}
\NormalTok{state.S.Dermatitis <-}\StringTok{ }\NormalTok{State}\OperatorTok{$}\KeywordTok{new}\NormalTok{(}
  \StringTok{"Dermatitis (Standard Dressing)"}\NormalTok{, }\DataTypeTok{cost=}\NormalTok{c.Dermatitis.S, }\DataTypeTok{interval=}\NormalTok{th}
\NormalTok{)}
\NormalTok{state.S.LSI <-}\StringTok{ }\NormalTok{State}\OperatorTok{$}\KeywordTok{new}\NormalTok{(}
  \StringTok{"Local site infection (Standard Dressing)"}\NormalTok{, }\DataTypeTok{cost=}\NormalTok{c.LSI.S, }\DataTypeTok{interval=}\NormalTok{th}
\NormalTok{)}
\NormalTok{state.S.CRBSI <-}\StringTok{ }\NormalTok{State}\OperatorTok{$}\KeywordTok{new}\NormalTok{(}
  \StringTok{"CRBSI (Standard Dressing)"}\NormalTok{, }\DataTypeTok{cost=}\NormalTok{c.CRBSI.S, }\DataTypeTok{interval=}\NormalTok{th}
\NormalTok{)}
\NormalTok{state.S.NoComp <-}\StringTok{ }\NormalTok{State}\OperatorTok{$}\KeywordTok{new}\NormalTok{(}
  \StringTok{"No complication (Standard Dressing)"}\NormalTok{, }\DataTypeTok{cost=}\NormalTok{c.S, }\DataTypeTok{interval=}\NormalTok{th}
\NormalTok{)}

\NormalTok{chance.S <-}\StringTok{ }\NormalTok{ChanceNode}\OperatorTok{$}\KeywordTok{new}\NormalTok{(}
  \DataTypeTok{children =} \KeywordTok{list}\NormalTok{(state.S.Dermatitis, state.S.LSI, state.S.CRBSI, state.S.NoComp),}
  \DataTypeTok{edgelabels =} \KeywordTok{c}\NormalTok{(}\StringTok{'Dermatitis'}\NormalTok{, }\StringTok{'Local site infection'}\NormalTok{, }\StringTok{'CRBSI'}\NormalTok{, }\StringTok{'No complication'}\NormalTok{),}
  \DataTypeTok{costs =} \KeywordTok{list}\NormalTok{(}\DecValTok{0}\NormalTok{, }\DecValTok{0}\NormalTok{, }\DecValTok{0}\NormalTok{, }\DecValTok{0}\NormalTok{),}
  \DataTypeTok{p =} \KeywordTok{list}\NormalTok{(p.Dermatitis.S, r.LSI, p.CRBSI.S, }\OtherTok{NA}\NormalTok{)}
\NormalTok{)}
\CommentTok{#> Warning: ChanceNode$new: 'ptype="MV"' may cause p values outside [0,1].}

\CommentTok{# Tegaderm dressing branch}
\NormalTok{state.T.Dermatitis <-}\StringTok{ }\NormalTok{State}\OperatorTok{$}\KeywordTok{new}\NormalTok{(}
  \StringTok{"Dermatitis (Tegaderm CHG)"}\NormalTok{, }\DataTypeTok{cost=}\NormalTok{c.Dermatitis.T, }\DataTypeTok{interval=}\NormalTok{th}
\NormalTok{)}
\NormalTok{state.T.LSI <-}\StringTok{ }\NormalTok{State}\OperatorTok{$}\KeywordTok{new}\NormalTok{(}
  \StringTok{"Local site infection (Tegaderm CHG)"}\NormalTok{, }\DataTypeTok{cost=}\NormalTok{c.LSI.T, }\DataTypeTok{interval=}\NormalTok{th}
\NormalTok{)}
\NormalTok{state.T.CRBSI <-}\StringTok{ }\NormalTok{State}\OperatorTok{$}\KeywordTok{new}\NormalTok{(}
  \StringTok{"CRBSI (Tegaderm CHG)"}\NormalTok{, }\DataTypeTok{cost=}\NormalTok{c.CRBSI.T, }\DataTypeTok{interval=}\NormalTok{th}
\NormalTok{)}
\NormalTok{state.T.NoComp <-}\StringTok{ }\NormalTok{State}\OperatorTok{$}\KeywordTok{new}\NormalTok{(}
  \StringTok{"No complication (Tegaderm CHG)"}\NormalTok{, }\DataTypeTok{cost=}\NormalTok{c.T, }\DataTypeTok{interval=}\NormalTok{th}
\NormalTok{)}

\NormalTok{chance.T <-}\StringTok{ }\NormalTok{ChanceNode}\OperatorTok{$}\KeywordTok{new}\NormalTok{(}
  \DataTypeTok{children =} \KeywordTok{list}\NormalTok{(state.T.Dermatitis, state.T.LSI, state.T.CRBSI, state.T.NoComp),}
  \DataTypeTok{edgelabels =} \KeywordTok{c}\NormalTok{(}\StringTok{'Dermatitis'}\NormalTok{, }\StringTok{'Local site infection'}\NormalTok{, }\StringTok{'CRBSI'}\NormalTok{, }\StringTok{'No complication'}\NormalTok{),}
  \DataTypeTok{costs =} \KeywordTok{list}\NormalTok{(}\DecValTok{0}\NormalTok{, }\DecValTok{0}\NormalTok{, }\DecValTok{0}\NormalTok{, }\DecValTok{0}\NormalTok{),}
  \DataTypeTok{p =} \KeywordTok{list}\NormalTok{(p.Dermatitis.T, r.LSI.T, p.CRBSI.T, }\OtherTok{NA}\NormalTok{)}
\NormalTok{)}
\CommentTok{#> Warning: ChanceNode$new: 'ptype="MV"' may cause p values outside [0,1].}

\CommentTok{# decision node}
\NormalTok{d <-}\StringTok{ }\NormalTok{DecisionNode}\OperatorTok{$}\KeywordTok{new}\NormalTok{(}
  \DataTypeTok{children =} \KeywordTok{list}\NormalTok{(chance.S, chance.T),}
  \DataTypeTok{edgelabels =} \KeywordTok{c}\NormalTok{(}\StringTok{'Standard'}\NormalTok{, }\StringTok{'Tegaderm'}\NormalTok{),}
  \DataTypeTok{costs =} \KeywordTok{list}\NormalTok{(}\DecValTok{0}\NormalTok{, }\DecValTok{0}\NormalTok{)}
\NormalTok{)}
\end{Highlighting}
\end{Shaded}

In the manufacturer's model, the uncertainties in the probabilities
associated with the polytomous chance nodes were modelled as independent
variables. This is not recommended because there is a chance that a
particular run of the PSA will yield probabilities that are outside the
range {[}0,1{]}. Representing the uncertain probabilities with draws
from a Dirichlet distribution is preferred. Creating a
\texttt{ChanceNode} with ModVars is permitted, but results in a warning
being issued.

\hypertarget{summary-of-the-model}{%
\subsection{Summary of the model}\label{summary-of-the-model}}

The model variables and their operands associated with a node and
(optionally) its descendants can be tabulated using the method
\texttt{tabulate\_modvars}. This returns a data frame describing each
variable, its description, units and uncertainty distribution. Variables
inheriting from type \texttt{ModVar} will be included in the tabulation;
regular numeric values will not be listed. For extensive models,
variables associated with separate branches of a tree can be tabulated
separately by calling the method for different head nodes.

The operands of model variables which are expressions of other model
variables can be included in the tabulation via the
\texttt{include.operands} parameter. This is recursive, allowing the
complete structure of a model, \emph{i.e.} its model variables and the
way in which they are combined, to be tabulated. In the Tegaderm model,
the complete structure is as follows:

\begin{longtable}[]{@{}lll@{}}
\toprule
Description & Label & Distribution\tabularnewline
\midrule
\endhead
No.~dressings & n.dressings & N(3,0.3)\tabularnewline
P(dermatitis\textbar{}standard dressing) & p.Dermatitis.S & n.dressings
* r.Dermatitis\tabularnewline
Baseline dermatitis risk & r.Dermatitis &
N(0.0026,0.00026)\tabularnewline
Baseline LSI rate & r.LSI & N(0.1,0.01)\tabularnewline
No.~days with catheter & n.cathdays & N(10,2)\tabularnewline
P(CRBSI\textbar{}standard dressing) & p.CRBSI.S & r.CRBSI *
n.cathdays/1000\tabularnewline
Baseline CRBSI rate & r.CRBSI & N(1.48,0.074)\tabularnewline
P(dermatitis\textbar{}Tegaderm) & p.Dermatitis.T & n.dressings *
r.Dermatitis * rr.Dermatitis\tabularnewline
Tegaderm Dermatitis RR & rr.Dermatitis & LN1(1.482,0.49)\tabularnewline
Tegaderm LSI HR & hr.LSI & LN1(-0.911,0.393)\tabularnewline
P(LSI\textbar{}Tegaderm) & r.LSI.T & r.LSI * hr.LSI\tabularnewline
Tegaderm CRBSI HR & hr.CRBSI & LN1(-0.911,0.393)\tabularnewline
P(CRBSI\textbar{}Tegaderm) & p.CRBSI.T & r.CRBSI * n.cathdays *
hr.CRBSI/1000\tabularnewline
\bottomrule
\end{longtable}

\hypertarget{point-estimates-and-distributions-of-model-variables}{%
\subsection{Point estimates and distributions of model
variables}\label{point-estimates-and-distributions-of-model-variables}}

The point estimates, units and distributional properties are obtained
from the same call, in the remaining columns. Rows with \texttt{Qhat}
indicate that the quantiles have been estimated from simulation.

\begin{longtable}[]{@{}llrrrl@{}}
\toprule
Description & Units & Mean & Q2.5 & Q97.5 & Qhat\tabularnewline
\midrule
\endhead
No.~dressings & dressings & 3.000 & 2.412 & 3.588 &\tabularnewline
P(dermatitis\textbar{}standard dressing) & P & 0.008 & 0.006 & 0.010 &
*\tabularnewline
Baseline dermatitis risk & /catheter & 0.003 & 0.002 & 0.003
&\tabularnewline
Baseline LSI rate & /patient & 0.100 & 0.080 & 0.120 &\tabularnewline
No.~days with catheter & days & 10.000 & 6.080 & 13.920 &\tabularnewline
P(CRBSI\textbar{}standard dressing) & P & 0.015 & 0.009 & 0.021 &
*\tabularnewline
Baseline CRBSI rate & /1000 catheter days & 1.480 & 1.335 & 1.625
&\tabularnewline
P(dermatitis\textbar{}Tegaderm) & P & 0.039 & 0.013 & 0.093 &
*\tabularnewline
Tegaderm Dermatitis RR & ratio & 4.963 & 1.685 & 11.500 &\tabularnewline
Tegaderm LSI HR & ratio & 0.434 & 0.186 & 0.869 &\tabularnewline
P(LSI\textbar{}Tegaderm) & P & 0.043 & 0.019 & 0.086 & *\tabularnewline
Tegaderm CRBSI HR & ratio & 0.434 & 0.186 & 0.869 &\tabularnewline
P(CRBSI\textbar{}Tegaderm) & P & 0.006 & 0.002 & 0.014 &
*\tabularnewline
\bottomrule
\end{longtable}

\hypertarget{running-the-model}{%
\subsection{Running the model}\label{running-the-model}}

The following code runs a single model scenario, using the
\texttt{evaluatePathways} method of a decision node to evaluate each
pathway from the decision node. In the model there are eight possible
root-to-leaf paths, each of which begins with the decision node and ends
with a leaf node. For example, pathway
\texttt{Dermatitis\ (Standard\ Dressing)} involves a traversal of nodes
\texttt{d}, \texttt{chance.S}, and \texttt{leaf.S.Dermatitis}. The
method \texttt{evaluateChoices} is similar, but aggregates the results
by choice. The results of the scenario model, using the code from the
previous section, yields the following table. This model did not
consider utility, and the columns associated with utility are removed.

\begin{longtable}[]{@{}llrrr@{}}
\toprule
Choice & Pathway & Probability & Cost & ExpectedCost\tabularnewline
\midrule
\endhead
Standard & Dermatitis (Standard Dressing) & 0.0078 & 154.02 &
1.20\tabularnewline
Standard & Local site infection (Standard Dressing) & 0.1000 & 254.02 &
25.40\tabularnewline
Standard & CRBSI (Standard Dressing) & 0.0148 & 9904.02 &
146.58\tabularnewline
Standard & No complication (Standard Dressing) & 0.8774 & 4.02 &
3.53\tabularnewline
Tegaderm & Dermatitis (Tegaderm CHG) & 0.0387 & 168.63 &
6.53\tabularnewline
Tegaderm & Local site infection (Tegaderm CHG) & 0.0434 & 268.63 &
11.67\tabularnewline
Tegaderm & CRBSI (Tegaderm CHG) & 0.0064 & 9918.63 &
63.77\tabularnewline
Tegaderm & No complication (Tegaderm CHG) & 0.9114 & 18.63 &
16.98\tabularnewline
\bottomrule
\end{longtable}

\hypertarget{model-results}{%
\subsection{Model results}\label{model-results}}

\hypertarget{base-case}{%
\subsubsection{Base case}\label{base-case}}

The total cost for each choice can be calculated from the table above,
or by calling \texttt{evaluateChoices}, giving a point estimate of the
saving of 77.76 GBP. This is close to the sponsor's point estimate of
cost saving estimated from their probabilistic sensitivity analysis,
77.26 GBP, reported in Jenks \emph{et al}\textsuperscript{1}.

\hypertarget{probabilistic-senstivity-analysis}{%
\subsubsection{Probabilistic senstivity
analysis}\label{probabilistic-senstivity-analysis}}

When they are created, each \texttt{ModVar} returns its expected value
when its method \texttt{value()} is called. Calling the method
\texttt{sample()} of a model variable causes it to sample from its
uncertainty distribution, and return the sampled value when method
\texttt{value()} is next called. The same sampled value will be returned
until \texttt{sample()} is called again. Calling
\texttt{sample(expected=T)} causes \texttt{value()} to return the
expected value of the variable.

Probabilistic sensitivity analysis is supported through the use of
sampling model variables. In practice, because the model variables enter
the model via nodes, the methods \texttt{evaluatePathways} and
\texttt{evaluateChoices} provided by decision nodes provide a convenient
interface for sampling from model variables. These methods, called with
\texttt{expected=FALSE} cause each model variable associated with the
decision node and its descendants to be sampled. For further convenience
the method \texttt{evaluateChoices} permits replicates to be run
(parameter \texttt{N}), making PSA straightforward. The first few runs
of PSA are as follows:

\begin{longtable}[]{@{}rrrr@{}}
\toprule
Run & Cost.Tegaderm & Cost.Standard & Difference\tabularnewline
\midrule
\endhead
1 & 141.79 & 183.51 & -41.71\tabularnewline
2 & 60.60 & 172.84 & -112.25\tabularnewline
3 & 167.82 & 193.26 & -25.44\tabularnewline
4 & 107.25 & 233.30 & -126.05\tabularnewline
5 & 119.51 & 184.56 & -65.05\tabularnewline
6 & 97.67 & 185.42 & -87.75\tabularnewline
7 & 97.53 & 181.76 & -84.22\tabularnewline
8 & 105.02 & 193.71 & -88.70\tabularnewline
9 & 89.34 & 130.00 & -40.65\tabularnewline
10 & 86.24 & 168.46 & -82.21\tabularnewline
\bottomrule
\end{longtable}

From PSA (1000 runs), the mean cost of treatment with Tegaderm was
98.38, the mean cost of treatment with standard dressings was 176.83 and
the mean cost saving was -78.45. The 95\% confidence interval for cost
saving was -137.9 to -15.17; the standard deviation of the cost saving
was 30.94. Overall, 98.7\% of runs found that Tegaderm was cost saving.
These results replicate those reported by the manufucturer (saving of
77.26, 98.5\% cases cost saving).

\hypertarget{an-alternative-tree-free-approach}{%
\section{An alternative, tree-free
approach}\label{an-alternative-tree-free-approach}}

It is possible to solve the decision problem without first constructing
a tree, by combining model variables directly. This is, in essence, the
approach taken in many decision tree models constructed in Excel.

\hypertarget{components-of-cost}{%
\subsection{Components of cost}\label{components-of-cost}}

Each cost component is defined as an expression involving two or more of
the 13 model inputs. In contrast to the tree approach, which computed
the cost of traversing each pathway, this approach allows the costs of
the technology and its comparator to be constructed as sub-totals.

\begin{Shaded}
\begin{Highlighting}[]
\CommentTok{# component costs, standard dressing}
\NormalTok{CHG.S <-}\StringTok{ }\NormalTok{ExprModVar}\OperatorTok{$}\KeywordTok{new}\NormalTok{(}
  \StringTok{"Cost std dressing"}\NormalTok{, }\StringTok{"GBP"}\NormalTok{, }\KeywordTok{quote}\NormalTok{(n.dressings}\OperatorTok{*}\NormalTok{c.Standard)}
\NormalTok{)}
\NormalTok{CRBSI.S <-}\StringTok{ }\NormalTok{ExprModVar}\OperatorTok{$}\KeywordTok{new}\NormalTok{(}
  \StringTok{"Cost CRBSI, std dressing"}\NormalTok{, }\StringTok{"GBP"}\NormalTok{, }\KeywordTok{quote}\NormalTok{(c.CRBSI}\OperatorTok{*}\NormalTok{r.CRBSI}\OperatorTok{*}\NormalTok{n.cathdays}\OperatorTok{/}\DecValTok{1000}\NormalTok{)}
\NormalTok{)}
\NormalTok{LSI.S <-}\StringTok{ }\NormalTok{ExprModVar}\OperatorTok{$}\KeywordTok{new}\NormalTok{(}
  \StringTok{"Cost LSI, std dressing"}\NormalTok{, }\StringTok{"GBP"}\NormalTok{, }\KeywordTok{quote}\NormalTok{(c.LSI}\OperatorTok{*}\NormalTok{r.LSI)}
\NormalTok{)}
\NormalTok{Dermatitis.S <-}\StringTok{ }\NormalTok{ExprModVar}\OperatorTok{$}\KeywordTok{new}\NormalTok{(}
  \StringTok{"Cost dermatitis, std dressing"}\NormalTok{, }\StringTok{"GBP"}\NormalTok{, }\KeywordTok{quote}\NormalTok{(r.Dermatitis}\OperatorTok{*}\NormalTok{c.Dermatitis}\OperatorTok{*}\NormalTok{n.dressings)}
\NormalTok{)}

\CommentTok{# component costs, Tegaderm}
\NormalTok{CHG.T <-}\StringTok{ }\NormalTok{ExprModVar}\OperatorTok{$}\KeywordTok{new}\NormalTok{(}
  \StringTok{"Cost Tegaderm"}\NormalTok{, }\StringTok{"GBP"}\NormalTok{, }\KeywordTok{quote}\NormalTok{(n.dressings}\OperatorTok{*}\NormalTok{c.Tegaderm)}
\NormalTok{)}
\NormalTok{CRBSI.T <-}\StringTok{ }\NormalTok{ExprModVar}\OperatorTok{$}\KeywordTok{new}\NormalTok{(}
  \StringTok{"Cost CRBSI, Tegaderm"}\NormalTok{, }\StringTok{"GBP"}\NormalTok{, }\KeywordTok{quote}\NormalTok{(c.CRBSI}\OperatorTok{*}\NormalTok{r.CRBSI}\OperatorTok{*}\NormalTok{hr.CRBSI}\OperatorTok{*}\NormalTok{n.cathdays}\OperatorTok{/}\DecValTok{1000}\NormalTok{)}
\NormalTok{)}
\NormalTok{LSI.T <-}\StringTok{ }\NormalTok{ExprModVar}\OperatorTok{$}\KeywordTok{new}\NormalTok{(}
  \StringTok{"Cost LSI, Tegaderm"}\NormalTok{, }\StringTok{"GBP"}\NormalTok{, }\KeywordTok{quote}\NormalTok{(c.LSI}\OperatorTok{*}\NormalTok{r.LSI}\OperatorTok{*}\NormalTok{hr.LSI)}
\NormalTok{)}
\NormalTok{Dermatitis.T <-}\StringTok{ }\NormalTok{ExprModVar}\OperatorTok{$}\KeywordTok{new}\NormalTok{(}
  \StringTok{"Cost dermatitis, Tegaderm"}\NormalTok{, }\StringTok{"GBP"}\NormalTok{, }
   \KeywordTok{quote}\NormalTok{(r.Dermatitis}\OperatorTok{*}\NormalTok{c.Dermatitis}\OperatorTok{*}\NormalTok{rr.Dermatitis}\OperatorTok{*}\NormalTok{n.dressings)}
\NormalTok{)}

\CommentTok{# per-patient costs}
\NormalTok{total.T <-}\StringTok{ }\NormalTok{ExprModVar}\OperatorTok{$}\KeywordTok{new}\NormalTok{(}
  \StringTok{'Treatment cost, Tegaderm'}\NormalTok{, }\StringTok{'GBP'}\NormalTok{, }\KeywordTok{quote}\NormalTok{(CHG.T}\OperatorTok{+}\NormalTok{CRBSI.T}\OperatorTok{+}\NormalTok{LSI.T}\OperatorTok{+}\NormalTok{Dermatitis.T)}
\NormalTok{)}
\NormalTok{total.S <-}\StringTok{ }\NormalTok{ExprModVar}\OperatorTok{$}\KeywordTok{new}\NormalTok{(}
  \StringTok{'Treatment cost, Standard'}\NormalTok{, }\StringTok{'GBP'}\NormalTok{, }\KeywordTok{quote}\NormalTok{(CHG.S}\OperatorTok{+}\NormalTok{CRBSI.S}\OperatorTok{+}\NormalTok{LSI.S}\OperatorTok{+}\NormalTok{Dermatitis.S)}
\NormalTok{)}
\NormalTok{c.diff <-}\StringTok{ }\NormalTok{ExprModVar}\OperatorTok{$}\KeywordTok{new}\NormalTok{(}
  \StringTok{'Cost difference'}\NormalTok{, }\StringTok{'GBP'}\NormalTok{, }\KeywordTok{quote}\NormalTok{(total.T}\OperatorTok{-}\NormalTok{total.S)}
\NormalTok{)}
\end{Highlighting}
\end{Shaded}

\hypertarget{base-case-1}{%
\subsection{Base case}\label{base-case-1}}

The components of cost can be extracted by tabulating the model
variables in the cost difference model variable, \texttt{c.diff} via
method \texttt{tabulate}. In this case, only the rows containing
component costs are displayed.

\begin{longtable}[]{@{}llrrrrl@{}}
\toprule
Description & Units & Mean & SD & Q2.5 & Q97.5 & Qhat\tabularnewline
\midrule
\endhead
Cost difference & GBP & -77.76 & 32.56 & -139.71 & -13.12 &
*\tabularnewline
Treatment cost, Tegaderm & GBP & 98.95 & 29.46 & 56.09 & 187.25 &
*\tabularnewline
Cost Tegaderm & GBP & 18.63 & 1.89 & 14.93 & 22.43 & *\tabularnewline
Cost CRBSI, Tegaderm & GBP & 63.65 & 28.74 & 22.11 & 124.15 &
*\tabularnewline
Cost LSI, Tegaderm & GBP & 10.86 & 4.77 & 4.57 & 22.72 &
*\tabularnewline
Cost dermatitis, Tegaderm & GBP & 5.81 & 3.53 & 1.78 & 14.89 &
*\tabularnewline
Treatment cost, Standard & GBP & 176.71 & 33.38 & 120.15 & 247.32 &
*\tabularnewline
Cost std dressing & GBP & 4.02 & 0.40 & 3.22 & 4.86 & *\tabularnewline
Cost CRBSI, std dressing & GBP & 146.52 & 31.94 & 85.54 & 214.82 &
*\tabularnewline
Cost LSI, std dressing & GBP & 25.00 & 4.13 & 16.83 & 33.59 &
*\tabularnewline
Cost dermatitis, std dressing & GBP & 1.17 & 0.26 & 0.73 & 1.74 &
*\tabularnewline
\bottomrule
\end{longtable}

The point estimate of saving is obtained directly from the expectation
of the cost difference variable, 77.76. This is identical to the value
obtained from the full tree. The table also indicates, by the quantiles
of the cost difference, the approximate confidence interval of the
saving.

\hypertarget{probabilistic-sensitivity-analysis}{%
\subsection{Probabilistic sensitivity
analysis}\label{probabilistic-sensitivity-analysis}}

Each model variable provides a method, \texttt{r}, to make random draws
from its uncertainty distribution. The PSA for the cost difference can
therefore be achieved by a single call to this method. From this, the
mean cost saving was -77.45, the 95\% confidence interval of the saving
was -138.37 to -13.36 and 99.1\% of runs generated a cost saving. Within
fluctuation error this is consistent with the manufacturer's reported
saving of 77.26 with 98.5\% of runs being cost saving.

\hypertarget{a-note-on-correlation-of-model-variables}{%
\section{A note on correlation of model
variables}\label{a-note-on-correlation-of-model-variables}}

Of the 13 input variables defined in the tree model, eight appear in
both the main branches of the decision tree. If, in PSA, common
variables are sampled from their uncertainty distributions once per run,
rather being sampled for each choice within a run, the costs of each arm
will be correlated.

In the original Excel model submitted by the manufacturer, the model
variables were sampled once per run. The tree model constructed using
\texttt{rdecision} described in this vignette made the same assumption,
and replicated the manufacturer's results. The correlation coefficient
between 1000 samples from the cost of treatment with Tegaderm and the
cost with standard care was 0.413. This reflects the relatively high
degree of correlation between the two choice arms of the model, because
of the eight shared model variables.

However, \texttt{rdecision} includes the facility to resample the model
variables for each choice, within each run, by setting parameter
\texttt{uncorrelate=TRUE} in method \texttt{evaluateChoices} of a
decision node. With this option, the equivalent correlation coefficient
is -0.046; \emph{i.e.} they are uncorrelated within statistical
fluctuation. This does not affect the point estimate, but does alter the
distribution of the cost difference, which has standard deviation 42.33
and 95\% confidence interval -153.22 to 5.44, with 96.8\% of simulations
being cost saving. The confidence interval includes zero, but the
percentage of cost saving cases exceeds 95\%; with correlation removed,
the evidence for a cost saving is therefore marginal, given the input
parameters.

Finally, the variance of the cost of treatment with Tegaderm is
30.51\(^2\), the variance of the cost of treatment with standard care is
29.92\(^2\) and the covariance with correlated model variables is
376.96. Via the relation \(Var(X-Y) = Var(X) + Var(Y) - 2Cov(X,Y)\), the
variance of the cost difference with correlated model variables is
32.75\(^2\) and with the uncorrelated model variables is 42.74\(^2\),
close to the observed values.

\hypertarget{references}{%
\section*{References}\label{references}}
\addcontentsline{toc}{section}{References}

\hypertarget{refs}{}
\leavevmode\hypertarget{ref-jenks:2016a}{}%
1. Jenks, M. \emph{et al.} Tegaderm CHG IV securement dressing for
central venous and arterial catheter insertion sites: A NICE Medical
Technology Guidance. \emph{Applied Health Economics and Health Policy}
\textbf{14}, 135--149 (2016).

\leavevmode\hypertarget{ref-briggs:2002a}{}%
2. Briggs, A., Claxton, K. \& Sculpher, M. \emph{Decision modelling for
health economic evaluation}. (Oxford University Press, 2006).

\end{document}
